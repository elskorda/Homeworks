\section{Εισαγωγή}

Η εξίσωση \textlatin{Schrondiger} που χρησιμοποιείται για  την περιγραφή των κβαντομηχανικών σωματιδίων προκύπτει από την μη σχετικιστική σχέση ενέργειας-ορμής $E^2=\frac{p^2}{2m}$ .  Όπως, είναι γνωστό, όμως, η σχέση αυτή δεν μπορεί να περιγράψει σωματίδια με μεγάλες ταχύτητες πολύ κοντά στην ταχύτητα του φωτός. Τα σωματίδια που μελετά η κβαντομηχανική στις περισσότερες περιπτώσεις μπορούν να φτάσουν αυτές της ταχύτητες, οπότε δημιουργείται η ανάγκη να ληφθεί υπόψη η σχέση ενέργειας-ορμής, της ειδικής θεωρίας της σχετικότητας $E^2=c^2p^2+m^2c^4$ .  Έτσι καταλήγουμε στην εξίσωση \textlatin{Klein-Gordon} ή οποία περιγράφει την κίνηση ενός σωματιδίου το οποίο δεν έχει σπιν. Πέρα από το ότι αδυνατεί να περιγράψει σωματίδια με σπιν, η εξίσωση \textlatin{Klein-Gordon} παρουσιάζει προβλήματα στην ερμηνεία της , όπως θα εξηγηθεί παρακάτω. Για τους παραπάνω λόγους ο \textlatin{Dirac} ακολούθησε διαφορετική μεθοδολογία και κατέληξε στην ομώνυμη εξίσωση, ή οποία περιλαμβάνει και το σπιν. 
Στα κεφάλαια που ακολουθούν θα δοθούν, αρχικά, οι ορισμοί βασικών εννοιών απαραίτητων για την επίλυση των ζητούμενων προβλημάτων.
Κατόπιν, αφού παρουσιαστεί η μορφή της εξίσωσης \textlatin{Klein-Gordon} για ελεύθερο σωματίδιο και σε περίπτωση παρουσίας διανυσματικού δυναμικού θα δειχτεί ότι για μια συνάρτηση που ικανοποιεί την εξίσωση \textlatin{Klein-Gordon} παρουσία διανυσματικού δυναμικού $A^\mu$, μ = 0,1,2,3, η εξίσωση συνέχειας ικανοποιείται από το τετραδιάνυσμα $J^\mu =\frac{i}{2m} \lbr \Psi* (\partial^\mu \Psi)-\Psi(\partial^\mu)* \rbr - \frac{e}{m}A^\mu \Psi* \Psi$.
Κατόπιν θα γίνει περιγραφή της εξίσωσης \textlatin{Dirac} και θα μελετηθεί το μη σχετικιστικό όριο αυτής αλλά και της \textlatin{Klein Gordon}

