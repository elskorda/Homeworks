\documentclass[12pt,oneside,titlepage,a4paper]{article}
\usepackage{epsfig,scrpage2,graphicx,float,hyperref}
\usepackage{caption,fullpage,amsmath,amssymb,mdframed}
\usepackage[greek,english]{babel}
\usepackage[utf8x]{inputenc}
\usepackage{subcaption}
\usepackage{multirow}
\setcounter{secnumdepth}{3}
\hypersetup{
  colorlinks=true,
  linkcolor=black,
  citecolor=black,
  urlcolor=blue!50!black
}
\captionsetup{
  labelfont=bf,
  font=small,
  format=hang 
}
%\setlength{\parindent}{0em}
%\setlength{\parskip}{0ex plus0.5ex minus0ex}
%\pagestyle{scrheadings}
\bibliographystyle{unsrt}   
%\oddsidemargin =-1cm
%\setlength{\textwidth}{7in}
%\addtolength{\voffset}{-5pt}
\renewcommand{\headfont}{\normalfont}

\cfoot{\pagemark}

\hypersetup{
  colorlinks=true,
  linkcolor=black,
  citecolor=black,
  urlcolor=blue!50!black
}

\newcommand{\rbr}{
  \ensuremath{\right) }
}

\newcommand{\lbr}{
  \ensuremath{\left( }
}

\newcommand{\rsbr}{
  \ensuremath{\right] }
}

\newcommand{\lsbr}{
  \ensuremath{\left[ }
}

\newcommand{\party}[2]{
  \ensuremath{\frac{\partial #1}{\partial #2}}
}
\newcommand{\deriv}[2]{
  \ensuremath{\frac{d #1}{d#2}}
}

\renewenvironment{abstract}[1][1.0]
{
	\begin{center}
		{\bf Περίληψη}\\[12pt]
		\begin{minipage}{#1\textwidth}
}
{
		\end{minipage}
	\end{center}
}

%------------------------------------------
%begin document 
%------------------------------------------

\begin{document}
%titlepage
\selectlanguage{greek}

\begin{titlepage}
	\begin{figure}[H]
		\centering
		\includegraphics[width = 4cm, keepaspectratio=1]{aristotleUniversityLogo.png}
	\end{figure}
	
	\begin{center}
		\large{{\sc Α.Π.Θ} ΣΧΟΛΗ ΘΕΤΙΚΩΝ ΕΠΙΣΤΗΜΩΝ ΤΜ. ΦΥΣΙΚΗΣ}\\[0.5cm]
		\LARGE\textbf{Εργασία για το μάθημα Σχετικιστική Κβαντομηχανική}\\[1.0cm] 

		\vspace{0.0cm}

		\small{Σκορδά Ελένη}\\[0.2cm]
               
		\small{Διδάσκων καθηγητής: Πασχάλης Ιωάννης }\\[0.2cm]

	\end{center}

	\begin{abstract}
	  Η παρούσα εργασία χωρίζεται σε δύο μέρη. Στο πρώτο μερος αποδεικνύεται ότι για μια συνάρτηση που ικανοποιεί την εξίσωση \textlatin{Klein-Gordon} παρουσία διανυσματικού δυναμικού $A^\mu$, μ = 0,1,2,3, η εξίσωση συνέχειας ικανοποιείται από το τετραδιάνυσμα $J^\mu =\frac{i}{2m} \lbr \Psi^* (\partial^\mu \Psi)-\Psi(\partial^\mu)\Psi^* \rbr - \frac{e}{m}A^\mu \Psi^* \Psi$ . Στο δευτερο μέλος της εργασίας μελετάται το μη σχετικιστικό όριο της εξίσωσης \textlatin{Dirac} και \textlatin{Klein Gordon}  
          
	\end{abstract}
        
	\vfill

	\centering{\footnotesize Τμήμα Φυσικής, Α.Π.Θ., \today}
\end{titlepage}

\newpage

% the table of contents is only updated when you run "pdflatex" twice 

\tableofcontents
\newpage


%-----------------------------------------
%eisagwgh
\section{Εισαγωγή}

Η εξίσωση \textlatin{Schr\"ondiger} που χρησιμοποιείται για  την περιγραφή των κβαντομηχανικών σωματιδίων προκύπτει με αντικατάσταση των τελεστών (\ref{energeiaormi}) στην μη σχετικιστική σχέση ενέργειας-ορμής $E^2=\frac{p^2}{2m}$  

\begin{align} 
  E & \rightarrow i\hbar \party{}{t}  &\vec{p} \rightarrow -i \hbar \nabla
  \label{energeiaormi}
\end{align} 

Οι λύσεις της εξίσωσης \textlatin{Schr\"ondiger}, είναι στην περίπτωση του ελεύθερου σωματιδίου, επίπεδα κύματα.

\[
  \Psi(\vec(r),t)=C e^{i(\vec(p)\vec(r)-Et)/\hbar}
\]

Όπως, είναι γνωστό, όμως, η σχέση αυτή δεν μπορεί να περιγράψει σωματίδια με μεγάλες ταχύτητες πολύ κοντά στην ταχύτητα του φωτός. Τα σωματίδια που μελετά η κβαντομηχανική στις περισσότερες περιπτώσεις μπορούν να φτάσουν αυτές τις ταχύτητες, οπότε δημιουργείται η ανάγκη να ληφθεί υπόψη η σχέση ενέργειας-ορμής, της ειδικής θεωρίας της σχετικότητας 

\begin{equation}
  E^2=c^2p^2+m^2c^4
  \label{genrel}
\end{equation} 

Έτσι καταλήγουμε στην εξίσωση \textlatin{Klein-Gordon} ή οποία περιγράφει την κίνηση ενός σωματιδίου το οποίο δεν έχει σπιν. Πέρα από το ότι αδυνατεί να περιγράψει σωματίδια με σπιν, η εξίσωση \textlatin{Klein-Gordon} παρουσιάζει προβλήματα στην ερμηνεία της , όπως θα εξηγηθεί παρακάτω. Για τους παραπάνω λόγους ο \textlatin{Dirac} ακολούθησε διαφορετική μεθοδολογία και κατέληξε στην ομώνυμη εξίσωση, ή οποία περιλαμβάνει και το σπιν. 


Στα κεφάλαια που ακολουθούν θα δοθούν, αρχικά, οι ορισμοί βασικών εννοιών απαραίτητων για την επίλυση των ζητούμενων προβλημάτων.
Κατόπιν, αφού παρουσιαστεί η μορφή της εξίσωσης \textlatin{Klein-Gordon} για ελεύθερο σωματίδιο και σε περίπτωση παρουσίας διανυσματικού δυναμικού θα δειχτεί ότι για μια συνάρτηση που ικανοποιεί την εξίσωση \textlatin{Klein-Gordon} παρουσία διανυσματικού δυναμικού $A^\mu$, μ = 0,1,2,3, η εξίσωση συνέχειας ικανοποιείται από το τετραδιάνυσμα $J^\mu =\frac{i}{2m} \lbr \Psi* (\partial^\mu \Psi)-\Psi(\partial^\mu)* \rbr - \frac{e}{m}A^\mu \Psi* \Psi$.
Κατόπιν θα γίνει περιγραφή της εξίσωσης \textlatin{Dirac} και θα μελετηθεί το μη σχετικιστικό όριο αυτής αλλά και της \textlatin{Klein Gordon}


\pagebreak
%-----------------------------------------
%xwroxronos Minkowski
\section{Ειδική Σχετικότητα}

Στο παρόν κεφάλαιο θα δοθούν, συνοπτικά, κάποιοι ορισμοί της ειδικής θεωρίας της σχετικότητας που χρησιμοποιούνται σε επόμενα κεφάλαια με έμφαση, όμως, στο  πως μετασχηματίζονται θεμελιώδεις έννοιες του Ηλεκτρομαγνητισμού στα πλαίσια της θεωρίας αυτής. Επίσης θα εξηγηθούν και ορισμένοι συμβολισμοί. 

\subsection{Τανυστές}


Βασικό μαθηματικό εργαλείο αποτελούν οι τανυστές, που είναι μία γενίκευση των βαθμωτών και διανυσματικών μεγεθών. Υπάρχουν τρία είδη τανυστών :

\begin{itemize}
\item οι συναλλοίωτοι: $ A'^{ij} = \party{x'_i }{x_k} \party{x'_j}{x_l} A^{kl} $
\item οι ανταλλοίωτοι :$ C'_{ij} = \party{x_k}{x'_i} \party{x_l}{x'_j}C_{kl}$ 
\item οι μεικτοί τανυστές: $ B'^i_j  =\party{x'_i }{x_k} \party{x_l}{x'_j} B^l_k  $
\end{itemize}

\flushleft 
Κάποιες βασικές και χρήσιμες ιδιότητες , που προκύπτουν σαν άμεση συνέπεια από τα παραπάνω είναι ότι :

\begin{itemize}
\item Αν ένα τανυστής μηδενίζεται σε ένα σύστημα συντεταγμένων , μηδενίζεται σε όλα 
\item Αν δύο τανυστές είναι ίσοι σε κάποιο σύστημα συντεταγμένων τότε θα ισούνται σε όλα 
\item Από τις δυο προηγούμενες ιδιότητες, προκύπτει ότι η μορφή μιας τανυστικής εξίσωσης δεν αλλάζει με την αλλαγή συστήματος 
\end{itemize}
	
Οι πράξεις ανάμεσα σε τανυστές είναι: 

\begin{itemize}
\item Άθροισμα : $C^{ab}_c = \alpha A^{ab}_c +\beta B^{ab}_c $ 
\item Εξωτερικό γινόμενο : ${C^{a}_{bcd}}^e = \alpha A^{a}_b +\beta {B_{ab}}^{e} $ 
\item Συστολή :$ {T_{aq}}^{cq} = \sum\limits_{q=1}^n {T_{aq}}^{cq} $, κατα συνεπεια ο τανυστής $ T^a_a = T $ είναι βαθμωτό μέγεθος.  \
\item Εσωτερικό γινόμενο 
\end{itemize}

\subsection{Χωρόχρονος \textlatin{Minkowski} }

Στην ειδική θεωρία της σχετικότητας ο χώρος δεν είναι ο τρισδιάστατος ευκλείδειος χώρος αλλά 
ο τετραδιάστατος χωρόχρονος \textlatin{Minkowski}. Ο μετρικός τανυστής για το χώρο αυτό ορίζεται ως 

\begin{equation}
  g_{\mu,\nu} =
  \begin{pmatrix}
    g_{0,0}  & \cdots & g_{0,3} \\
    \vdots & \ddots & \vdots \\
    g_{3,0}  & \cdots & g_{3,3}
  \end{pmatrix}
  =
  \begin{pmatrix}
    1 & 0 & 0 & 0 \\
    0 & -1 & 0 & 0 \\
    0 & 0 & -1 & 0 \\
    0 & 0 & 0 &-1
  \end{pmatrix}
\end{equation}

Ο πίνακας που δίνει το  μετασχηματισμό ενός οποιουδήποτε τετραδιανύσματος ανάμεσα σε δύο αδρανειακά συστήματα, (μετασχηματισμοί \textlatin{Lorentz}) είναι : 

\begin{equation}
  \Lambda_\mu ^\nu=
  \begin{pmatrix}
    \Lambda^0 _0  & \cdots & \Lambda^0 _3 \\
    \vdots & \ddots & \vdots \\
    \Lambda^3 _0  & \cdots & \Lambda^3 _3
  \end{pmatrix}
  =
  \begin{pmatrix}
    \gamma & -\gamma \beta& 0 & 0 \\
    -\gamma \beta & \gamma & 0 & 0 \\
    0 & 0 & 1 & 0 \\
    0 & 0 & 0 & 1
  \end{pmatrix} 
\end{equation}



\pagebreak
%-----------------------------------------
%eksiswsi KleinGordon
\section{Εξίσωση \textlatin{Klein-Gordon}}

\subsection{Εξίσωση \textlatin{Klein-Gordon} για ελεύθερο σωματίδιο}
Για τον υπολογισμό της εξίσωσης \textlatin{Klein-Gordon} αρκεί να αντικατασταθούν οι τελεστές ενέργειας και ορμής (\ref{energeiaormi}) στην (\ref{genrel}): 

\begin{align}
 \notag (i\hbar \party{}{t} )^2 &= (-i\hbar \nabla)^2 c^2 +m_0 ^2 c^4
  \\ -\hbar^2 \party{^2}{t^2} &= -\hbar^2 \nabla^2 c^2 +m_0^2 c^4 
  \label{kg1} 
\end{align}

Για $\hbar=c=1$: 

\[
\lbr \party{^2}{t^2}-\nabla^2 \rbr \Psi +m_0^2 \Psi = 0
\]

Ο πρώτος όρος του αριστερού μέρους είναι η γνωστη ντ'αλαμπερσιανή $\Box = (\party{^2}{t^2}-\nabla^2) $. Επομένως η εξίσωση παίρνει την τελική μορφή: 

\[
  \Box \Psi +m_0^2 \Psi = 0  
\]

Η εξίσωση (\ref{kg1}) μπορεί να γραφτεί και με τανυστική μορφή, αν στην(\ref{genrel2})  αντικατασταθεί το τετραδιάνυσμα της ορμής. Ισχύει ότι : 

\[ 
{p}^\mu {p}_\mu=-\hbar^2 \party{}{x^\mu}\party{}{x_\mu}=\frac{1}{c^2} \party{^2}{t^2} -\nabla
\]

Επομένως η εξίσωση \textlatin{Klein-Gordon} για ελεύθερο σωματίδιο και για $\hbar = c = 1 $ γράφεται: 

\begin{equation}
  {p}^\mu {p}_\mu \Psi +m_0^2 \Psi = 0
  \label{kg2}
\end{equation}
%----------------------------------------
%    new subsection
%----------------------------------------
\subsection{Εξίσωση συνέχειας και τετραδιάνυσμα ρεύματος για ελεύθερο σωματίδιο}  

Για να υπολογιστεί η μορφή του τετραδιάνυσμα του ρεύματος, υπολογίζεται αρχικά το συζυγές της (\ref{kg1}) :

\[
\lbr \hbar^2 \party{^2}{t^2}- \hbar^2 c^2 \nabla^2 \rbr \Psi^* +m_0^2 c^4 \Psi^* = 0
\]

Κατόπιν η παραπάνω σχέση πολλαπλασιάζεται από τα αριστερά με $\Psi$ και η (\ref{kg1}) από τα αριστερά, πάλι, με $\Psi^*$. Η διαφορά των σχέσεων σχέσεων αυτών δίνει: 

\begin{align*}
  \Psi^* \lbr  \hbar^2 \party{^2}{t^2}- \hbar^2 c^2 \nabla^2 \rbr \Psi +\Psi^* m_0^2 \Psi - \Psi \lbr \hbar^2\party{^2}{t^2}- \hbar^2 c^2 \nabla^2 \rbr \Psi^* -\Psi m_0^2 c^4 \Psi^* &= 0 \Rightarrow
  \\ \hbar^2\Psi^* \party{^2}{t^2} \Psi - \hbar^2 c^2\Psi^* \nabla^2\Psi +\Psi^* c^4 m_0^2 \Psi - \Psi \hbar^2\party{^2}{t^2}\Psi^* + \hbar^2 c^2 \Psi \nabla^2 \Psi^* + \Psi c^4  m_0^2 \Psi^* &= 0 \Rightarrow
\end{align*}

\[
 \frac{1}{c^2}\Psi^* \party{^2}{t^2} \Psi - \Psi^* \nabla^2\Psi  - \frac{1}{c^2} \Psi \party{^2}{t^2}\Psi^* + \Psi \nabla^2 \Psi^* = 0 \Rightarrow
\]

\[
 \frac{1}{c^2} \lbr \Psi^* \party{^2}{t^2} \Psi - \Psi \party{^2}{t^2}\Psi^* \rbr = \Psi^* \nabla^2\Psi- \Psi \nabla^2 \Psi^*  \Rightarrow
\]

\[ 
\frac{1}{c^2} \lbr \Psi^* \party{^2}{t^2} \Psi + \party{\Psi^*}{t} \party{\Psi}{t}-\party{\Psi^*}{t} \party{\Psi}{t}- \Psi \party{^2}{t^2}\Psi^* \rbr = \Psi^* \nabla^2\Psi- \Psi \nabla^2 \Psi^*  \Rightarrow
\]

\[
 \frac{1}{c^2} \lsbr \party{}{t} \lbr \Psi^* \party{\Psi}{t} - \Psi\party{\Psi^*}{t} \rbr \rsbr = \Psi^* \nabla^2\Psi- \Psi \nabla^2 \Psi^*  
\]

\[
\frac{1}{c^2} \lsbr \party{}{t} \lbr \Psi^* \party{\Psi}{t} - \Psi\party{\Psi^*}{t} \rbr \rsbr = \Psi^* \nabla^2\Psi- \Psi \nabla^2 \Psi^*+\nabla \Psi^* \nabla \Psi-  \nabla \Psi^* \nabla \Psi  
\]

\begin{equation}
 \frac{1}{c^2} \lsbr \party{}{t} \lbr \Psi^* \party{\Psi}{t} - \Psi\party{\Psi^*}{t} \rbr \rsbr = \nabla \lbr \Psi^* \nabla \Psi- \Psi \nabla \Psi^* \rbr  
 \label{fineqfree}
\end{equation}

Συγκρίνοντας την εξίσωση συνέχειας (\ref{contineq}) με την προηγούμενη προκύπτει: 

\begin{align}
  \notag \rho &= \frac{1}{c^2} \lbr \Psi^* \party{\Psi}{t} - \Psi\party{\Psi^*}{t} \rbr   &\vec{J}= \Psi^* \nabla \Psi- \Psi \nabla \Psi^*   \Rightarrow
  \\  \rho &= \frac{i\hbar}{2mc^2} \lbr \Psi^* \party{\Psi}{t} - \Psi\party{\Psi^*}{t} \rbr   &\vec{J}= \frac{i\hbar}{2m}\Psi^* \nabla \Psi- \Psi \nabla \Psi^*   
  \label{currentfree}
\end{align}

 Kαι τα τα δύο μέλη της (\ref{fineqfree}) πολλαπλασιάστηκαν με $\frac{i \hbar}{2m}$ ώστε το $\rho$ της (\ref{currentfree}) να έχει διαστάσεις πυκνότητας πιθανότητας.
%----------------------------------------
%    new subsection
%----------------------------------------
\subsection{Εξίσωση \textlatin{Klein-Gordon} παρουσία ηλεκτρομαγνητικού πεδίου}

Στην περίπτωση που υπάρχει παρουσία διανυσματικού πεδίου, στους τελεστές ενέργειας και ορμής (\ref{energeiaormi}) εισάγονται τα δυναμικά $\vec{A}$ και $\phi$ έτσι ώστε ο τρόπος εκλογής τους να μην έχει φυσική σημασία. Η διαδικασία αυτή γνωστή και ως ελάσσονα αντικατάσταση: 

\begin{align} 
  E & \rightarrow i\hbar \party{}{t} -q\phi  &\vec{p} \rightarrow -i \hbar \nabla-q\vec{A}
  \label{energeiaormi2}
\end{align} 

Με τανυστική μορφή μπορεί ο παραπάνω μετασχηματισμός γράφεται : 

\begin{align}
  p^\mu & \rightarrow p^\mu-qA^\mu,    &p_\mu \rightarrow p_\mu-qA_\mu
  \label{tensormin}
\end{align}

Η εξίσωση για το ελεύθερο σωματίδιο, σύμφωνα με τα παραπάνω , μετασχηματίζεται παρουσία πεδιου(για $\hbar=c=1$, και γράφεται ως εξής : 

\[
\lbr i \party{}{t}-q\phi \rbr ^2 \Psi = \lbr -i \nabla -q\vec{A} \rbr ^2 +m_0^2 
\label{kgfield} 
\]

Η τανυστική μορφή της παραπάνω εξίσωσης, προκύπτει αν στην (\ref{kg2}) αντικατασταθεί η (\ref{tensormin}): 

\begin{align}
  \notag &\lbr p^\mu-\frac{q}{c}A^\mu \rbr \lbr {p}_\mu-\frac{q}{c}A_\mu \rbr \Psi +m_0^2 c^2 \Psi = 0 \Rightarrow
  \\  &\lsbr g^{\mu\nu} \lbr i \hbar\party{}{x^\nu}-\frac{q}{c}A_\nu \rbr \lbr i \hbar\party{}{x^\mu}-\frac{q}{c}A_\mu \rbr\rsbr \Psi+m_0^2 c^2 \Psi  =0 
  \label{eqn1}
\end{align}

\subsection{Εξίσωση συνέχειας και τετραδιάνυσμα ρεύματος παρουσία διανυσματικου δυναμικού $A^\mu$}  

Για την μελέτη των πυκνοτήτων ρεύματος και φορτίου , ακολουθείται παρόμοια  διαδικασία  με της παραγράφου 3.2. Συγκεκριμένα, η \ref{eqn1} θα πολλαπλασιαστεί από τα αριστερά με $\Psi^*$ και κατόπιν υπολογίζεται το συζυγές της. 

\begin{align}
 \notag &\Psi^* \lsbr g^{\mu\nu} \lbr i \hbar\party{}{x^\nu}-\frac{q}{c}A_\nu \rbr \lbr i \hbar\party{}{x^\mu}-\frac{q}{c}A_\mu \rbr\rsbr \Psi + \Psi^* m_0^2 c^2 \Psi=0 &\Rightarrow
 \\\notag  &\left\{ \Psi^* \lsbr g^{\mu\nu} \lbr i \hbar\party{}{x^\nu}-\frac{q}{c}A_\nu \rbr \lbr i \hbar\party{}{x^\mu}-\frac{q}{c}A_\mu \rbr\rsbr \Psi + \Psi^* m_0^2 c^2 \Psi \right\} ^*=0 &\Rightarrow
 \\  &\Psi \lsbr g^{\mu\nu} \lbr -i \hbar\party{}{x^\nu}-\frac{q}{c}A_\nu \rbr \lbr -i \hbar\party{}{x^\mu}-\frac{q}{c}A_\mu \rbr\rsbr \Psi^* + \Psi m_0^2 c^2 \Psi^*=0 
 \label{eqn2}
\end{align}

Η (\ref{eqn1}) πολλαπλασιάζεται από αριστερά με $\Psi^*$ και , από αυτό που θα προκύψει, αφαιρείται από την (\ref{eqn2}): 


\begin{align*}
  &\Psi \lsbr g^{\mu\nu} \lbr -i \hbar\party{}{x^\nu}-\frac{q}{c}A_\nu \rbr \lbr -i \hbar\party{}{x^\mu}-\frac{q}{c}A_\mu \rbr\rsbr \Psi^* -
  \\&- \Psi^* \lsbr g^{\mu\nu} \lbr i \hbar\party{}{x^\nu}+\frac{q}{c}A_\nu \rbr \lbr i \hbar\party{}{x^\mu}+\frac{q}{c}A_\mu \rbr\rsbr \Psi =0 &\Rightarrow 
  \\& \Psi^* \lsbr -g^{\mu\nu} \lbr i \hbar\party{}{x^\nu}+\frac{q}{c}A_\nu \rbr \lbr i \hbar\party{}{x^\mu}+\frac{q}{c}A_\mu \rbr\rsbr \Psi- 
  \\&-\Psi \lsbr - g^{\mu\nu} \lbr -i \hbar\party{}{x^\nu}-\frac{q}{c}A_\nu \rbr \lbr -i \hbar\party{}{x^\mu}-\frac{q}{c}A_\mu \rbr\rsbr \Psi^* =0 &\Rightarrow  
\end{align*}

\pagebreak
%-----------------------------------------
%eksiswsi KleinGordon
\section{Μη σχετικιστικό όριο εξίσωσης \textlatin{Klein-Gordon}}

Στο μη σχετικιστικό όριο η ταχύτητα του σωματιδίου είναι πολύ μικρή σε σχέση με την ταχύτητα του φωτός στο κενό $(\upsilon<<C)$ οπότε θα ισχύει και $p<<mc$.Η σχέση (\ref{genrel}) γίνεται: 

\[
  E=\lbr c^2p^2+m_0^2c^4 \rbr^{1/2} =mc^2 \lbr 1+ \frac{p^2}{m_0^2c^2} \rbr^{1/2} \approx mc^2 \lbr 1+ \frac{1}{2} \frac{p^2}{m_0^2c^2} \rbr \Rightarrow
\]

\begin{equation}
  E \approx  m_0 c^2 + \frac{p^2}{2m} \Rightarrow E'=E - m_0 c^2
  \label{nrlim}
\end{equation}

Από την σχέση (\ref{nrlim}) παρατηρείται ότι στην οριακή περίπτωση η συνολική ενέργεια του σωματιδίου διαφέρει ελάχιστα από την ενέργεια ηρεμίας. Στις επόμενες παραγράφους θα μελετηθεί το μη σχετικιστικό όριο για την περίπτωση ελεύθερου σωματιδίου όσο και για την περίπτωση που υπάρχει παρουσία ηλεκτρομαγνητικού πεδίου.

%============================================================================
%                             new subsection
%============================================================================
\subsection{Μη σχετικιστικό όριο ελεύθερης εξίσωσης \textlatin{Klein-Gordon}}

Για να μελετηθεί το μη σχετικιστικό όριο, η λύση της εξίσωσης γράφεται με την παρακάτω χωριζόμενη μορφή,με αντικατάσταση της(\ref{nrlim}) στην(\ref{sol1})


\begin{equation}
  \Psi(\vec{r},t) =e^{ \frac{i}{\hbar}\lbr \vec{p}\vec{x} -\lbr m_0 c^2 + \frac{p^2}{2m} \rbr t/\hbar \rbr }=e^{ \frac{i}{\hbar}\lbr \vec{p}\vec{x} -E't \rbr} e^{-im_0c^2t/\hbar }= \phi(\vec{r},t)e^{-im_0c^2t/\hbar } 
  \label{soludist}
\end{equation}

Στην ουσία η εξάρτηση από τον χρόνο χωρίστηκε σε δύο όρους: ο ένας από τους οποίους περιέχει την μάζα ηρεμίας του σωματιδίου και ο άλλος είναι ακριβώς η λύση της ελεύθερης εξίσωσης \textlatin{Schr\"ondiger} με χαμιλτονιανή την $H=p^2/2m_0$.Επομένως θα ισχύει: 

\begin{equation} 
  \left| i\hbar\party{\phi}{t} \right| \approx E' \phi << m_0 c^2\phi
  \label{eqnsometh1}
\end{equation}

Αρχικά θα υπολογιστουν η πρώτη και δεύτερη παράγωγος ως προς το χρόνο της Ψ και θα γινει αντικατάσταση στη (\ref{kg2}).

\begin{equation}
  \begin{split}
  \party{\Psi}{t} &= \party{}{t} \lbr \phi e^{-im_0c^2t/\hbar } \rbr = \party{\phi}{t} e^{-im_0c^2t/\hbar }  -im_oc^2/\hbar \phi e^{-im_0c^2t/\hbar }=
  \\&=\lbr \party{\phi}{t}   -im_oc^2/\hbar \phi\rbr e^{-im_0c^2t/\hbar }   \stackrel{(\ref{eqnsometh1})}{\approx}  -im_oc^2/\hbar \phi e^{-im_0c^2t/\hbar }
  \end{split}
  \label{somet1}
\end{equation}

\begin{equation}
  \begin{split}
  \party{^2 \Psi}{t^2} &= \party{}{t} \lsbr \lbr \party{\phi}{t}-im_oc^2/\hbar \phi \rbr e^{-im_0c^2t/\hbar }  \rsbr=
  \\&=\lbr \party{^2\phi}{t^2}-im_0c^2\party{\phi}{t}/\hbar -im_0c^2\party{\phi}{t}/\hbar- m_0^2c^4\phi/\hbar \rbr e^{-im_0c^2t/\hbar } =
  \\& \approx -\lbr 2im_0c^2\party{\phi}{t}/\hbar+ m_0^2c^4\phi/\hbar \rbr e^{-im_0c^2t/\hbar } 
  \end{split}
  \label{somet2}
\end{equation}

Στην εξίσωση (\ref{somet2}), η δευτερη παράγωγος της φ ως προς το χρονο θεωρείται αμελητέα. Τέλος η (\ref{somet2}) εισάγεται στην εξίσωση \textlatin{Klein-Gordon} οπότε : 

\[
-\frac{1}{c^2}\lbr 2im_0c^2\party{\phi}{t}/\hbar+ m_0^2c^4\phi/\hbar \rbr e^{-im_0c^2t/\hbar }  = \lbr \party{^2}{x^2}+\party{^2}{y^2}+\party{^2}{z^2}- \frac{m_0^2c^2 }{\hbar}\rbr \phi e^{-im_0c^2t/\hbar }
\]
\vspace{0.5cm}
\begin{mdframed}
  \[   
  i\hbar\party{\phi}{t}  = -\frac{\hbar^2}{2m_0}\nabla^2 \phi
  \]
\end{mdframed}
\vspace{0.5cm}
Η τελευταία εξίσωση δεν είναι άλλη απο την ελεύθερη εξίσωση του \textlatin{Schr\"ondiger} για σωματίδια χωρίς σπιν.
%============================================================================
%                             new subsection
%============================================================================
\subsection{Μη σχετικιστικό όριο εξίσωσης \textlatin{Klein-Gordon} παρουσία μαγνητικού πεδίου}
 
Και σε αυτή τη  περίπτωση ακολουθείται η ίδια διαδικασία με της προηγούμενης παραγράφου.Η λύση γράφεται με την μορφή της εξίσωσης (\ref{soludist}). Η φ και σε αυτή την περίπτωση αποτελεί το μη σχετικιστικό κομμάτι της λύσης για το οποίο θα πρέπει να ισχύουν οι παρακάτω σχέσεις : 

\begin{align}
 & \left| i\hbar\party{\phi}{t} \right|  << m_0 c^2\phi , & \left| eA_0 \phi \right| << m_0 c^2|\phi|
  \label{somet3}
\end{align}

Στην (\ref{somet3}) $A_0$ είναι το βαθμωτο δυναμικο φ. Χρησιμοποιείται ο συμβολισμός αυτος για να μην υπάρχει σύγχυση με την κυματοσυνάρτηση. Η πρώτη σχέση εκφράζει το πόσο μικρή είναι η μη σχετικιστική ενέργεια σε σχέση με την ενέργεια ηρεμίας (όπως και στη προηγούμενη παράγραφο), ενώ η δευτερη ότι το δυναμικό πρεπει να είναι επίπεδο σε σύγκριση με την ενέργεια ηρεμίας, αλλιώς είναι αδύνατη η μελέτη του μη σχετικιστικού ορίου.  

\begin{align*}
  &\lbr i \hbar \party{}{t}-q A_0 \rbr \Psi = \lbr i \hbar \party{}{t}-q A_0 \rbr \phi e^{-im_0c^2t/\hbar }=\lbr i \hbar \party{\phi}{t} + m_0c^2 \phi -q A_0  \phi\rbr e^{-im_0c^2t/\hbar }
  \\
  \\& \lbr i \hbar \party{}{t}-q A_0 \rbr^2 \Psi = \lbr i \hbar \party{}{t}-q A_0 \rbr \lbr i \hbar \party{\phi}{t} + m_0c^2 \phi -q A_0  \phi\rbr e^{-im_0c^2t/\hbar }=
  \\&= i \hbar \party{}{t} \lbr i \hbar \party{\phi}{t} + m_0c^2t /phi -q A_0  \phi\rbr e^{-im_0c^2t/\hbar }-q A_0\lbr i \hbar \party{\phi}{t} + m_0c^2 \phi -q A_0  \phi\rbr e^{-im_0c^2t/\hbar } =
  \\&=\lbr -\hbar^2 \party{^2\phi}{t^2}- i \hbar \party{ A_0 }{t}\phi -i\hbar q A_0 \party{\phi}{t}+i\hbar m_0c^2\party{\phi}{t}\rbr e^{-im_0c^2t/\hbar }+
  \\&+ \lbr-i\hbar q \party{\phi}{t}A_0+qA_0^2\phi-q A_0\phi m_0 c^2 \phi+i\hbar m_0 c^2 \party{\phi}{t}-q A_0 \phi m_0 c^2 m_0^2 c^4\rbr  e^{-im_0c^2t/\hbar }
  \\& \approx e^{-im_0c^2 t/\hbar } \lbr m_0^2c^2-2m_0 q A_0 +2 m_0 c^2 i \hbar \party{}{t} -i\hbar q \party{A_0}{t} \rbr \phi
\end{align*}

Το αποτέλεσμα αυτό αντικαθιστάται στην εξίσωση (\ref{kgfield}) οπότε: 

\[
  \lbr m_0^2c^2-2m_0 q A_0 +2 m_0 c^2 i \hbar \party{}{t} -i\hbar q \party{A_0}{t} \rbr e^{-im_0c^2 t/\hbar } \phi =e^{-im_0c^2 t/\hbar } \lsbr \lbr +i\hbar \nabla +\frac{q}{c} A\rbr ^2 +m_0^2 c^2 \rsbr \phi
\]
\vspace{0.5cm}
\begin{mdframed}
  \[
   i\hbar \party{\phi}{t} =\frac{1}{2 m_0}\lbr +i\hbar \nabla +\frac{q}{c} A\rbr ^2 +q A_0+\frac{i\hbar q}{2m_0 c^2} \party{A_0}{t} 
   \]
\end{mdframed}
\vspace{0.5cm}
Η παραπάνω εξίσωση ειναι η εξίσωση του \textlatin{Schr\"ondiger} για ηλεκτρομαγνητικό δυναμικό. Παρατηρείται,επομένως ότι στο μη σχετικιστικό όριο η εξίσωση \textlatin{Klein-Gordon} ταυτίζεται με την εξίσωση του \textlatin{Schr\"ondiger} και στις δύο περιπτώσεις που μελετήθηκαν, για σωματίδια χωρίς σπίν.

\pagebreak
%-----------------------------------------
%eksiswsi KleinGordon
%====================================================
%               N E W   S E C T I O N  
%====================================================
\section{Μη σχετικιστικό όριο εξίσωσης \textlatin{Dirac}}
Είναι σημαντικό να εξεταστεί όπως έγινε στην περίπτωση της εξίσωσης \textlatin{Klein-Gordon}, η συμπεριφορά της εξίσωσης, στο όριο των χαμηλών ταχυτήτων. Θα εξετασθούν οι περιπτώσεις της μονοδιάστατης εξίσωσης και της εξίσωσης παρουσίας ηλεκτρομαγνητικού πεδίου. Στην πρώτη περίπτωση αναμένεται η εξίσωση να προσεγγίζει την εξίσωση \textlatin{Schr\"ondiger} ενώ στη δεύτερη την εξίσωση του \textlatin{Pauli}. 
%==============  new subsection   ============================= 
\subsection{Ελεύθερη κίνηση - περίπτωση μονοδιάστατης εξίσωσης \textlatin{Dirac}-μη σχετικιστικό όριο}
\subsubsection{Ελεύθερη κίνηση}
Η μελέτη θα ξεκινησει από την εξίσωση(\ref{somesoldir}). Στο μη σχετικιστικό όριο και για την περίπτωση των θετικών ενεργειών θα ισχύει:\\
\[ E_p \approx m_0c^2 + \frac{p^2}{2m_0} \]\\
Από την (\ref{somesoldir}) θα ισχύει ότι: \\
\begin{equation}
  \chi_0 =\frac{c \sigma \cdot p }{2m_0 c^2 + \frac{p^2}{2m_0}}\phi_0
\end{equation} \\
Επειδή $2m_0 c^2 >> \frac{p^2}{2m_0}$ θα ισχύει ότι $\phi_0 >> \chi_0 $. Επομένως η συνιστώσα $chi_0 $ μπορεί να θεωρηθεί αμελητέα. Οι λύσεις στο όριο αυτό θα έχουν τη μορφή: 

\begin{align*} 
  \Psi = 
  \begin{pmatrix}
    &1\\ &0\\ &0\\ &0  
  \end{pmatrix}
  e^{ipr/ \hbar},
  &\Psi = 
  \begin{pmatrix}
    &0\\ &1\\ &0\\ &0  
  \end{pmatrix}
  e^{ipr/ \hbar}
\end{align*}\\
Οι ιδιοσυναρτήσεις σ'αυτη τη περίπτωση, που η χ συνιστώστα δεν λαμβάνεται υπόψη, αντιστοιχούν στο σπίν του σωματιδίου και έχουν ιδιοτιμές $+1/2, -1/2$ αντίστοιχα
\subsubsection{Μονοδιάστατη κίνηση}
Στην παράγραφο αυτή θα παρουσιαστεί η περίπτωση της μονοδιάστατης εξίσωσης \textlatin{Dirac}, που αποτελεί ειδικότερη περίπτωση της τρισδιάστατης εξίσωσης που παρουσιάστηκε στην προηγούμενη παράγραφο.\\ 
Η μονοδιάστατη εξίσωση είναι : 
\[
\begin{pmatrix}
  &m &p\\
  &p &-m
\end{pmatrix}
\cdot
\begin{pmatrix}
  &\phi\\
  &\chi
\end{pmatrix}
= E \cdot
\begin{pmatrix}
  &\phi\\
  &\chi
\end{pmatrix}
\]
\\
Από την οποία προκύπτουν οι εξισώσεις : \\
\begin{align*}
  &m\phi+p\chi=E\phi
  \\&p\phi-m\chi=E\chi
\end{align*}\\
Από αυτές λύνοντας ως προς χ ισχύει $ \chi =\frac{p\phi}{E+m}$. Στο μη σχετικιστικό όριο η κινητική ενέργεια του σωματιδίου είναι πολύ μικρή σε σχέση με την ενέργεια ηρεμίας, οπότε ισχύει  $E \approx m$ άρα $ \chi =\frac{p\phi}{2m}$. Η αντικατάσταση του αποτελέσματος αυτού στην πρώτη από τις προηγούμενες εξισώσεις δίνει :\\
\[ m\phi +\frac{p^2}{2m}\phi=E\phi \]\\
Η παραπάνω σχέση δεν είναι άλλη από την εξίσωση του \textlatin{Schr\"ondiger} για ελεύθερο σωματίδιο.\\
%==============  new subsection   =============================  
\subsection{Mη σχετικιστικό όριο παρουσία ηλεκτρομαγνητικού πεδίου}
Η μελέτη του ορίου θα ξεκινήσει, απο την εξίσωση (\ref{diremfinal}). Αρχικά θα μελετηθεί η :\\ 
\begin{equation}
  i\hbar \party{\chi}{t} = c \sigma \cdot \hat{\Pi} \phi+ e A_0\chi -2m_0 c^2 \chi
  \label{nonreldir1}
\end{equation}\\
Στο συγκεκριμένο όριο η κινητική ενέργεια του σωματιδίου, όπως και η δυναμική ενεργεια, είναι μικρές σε σύγκριση με την ενέργεια ηρεμίας.Αυτό επιτρέπει να γίνουν οι προσσεγγισεις\\ 
\begin{align*}
  &\left| i \hbar \party{}{t}\right| << \left| m_0 c^2 \chi \right|, &\left| e A_0 \chi \right| << \left| m_0 c^2 \chi \right|
\end{align*}\\
Από την (\ref{nonreldir1}) προκύπτει: \\
\[
  i\hbar \party{\chi}{t}+2m_0 c^2 \chi- e A_0\chi = c \sigma \cdot \hat{\Pi} \phi \Rightarrow
\]\\
\begin{equation}
  \chi = \frac{\sigma \cdot\hat{\Pi} }{2m_0 c} \phi
  \label{nonreldir2}
\end{equation}\\
Η εισαγωγή της (\ref{nonreldir2}) στην (\ref{nonreldir1}) θα οδηγήσει στην μη σχετικιστικη έκφραση της κυματοσυνάρτησης $\phi$ : \\
\[
 i\hbar \party{\phi}{t} = c \sigma \cdot \hat{\Pi} \chi+ e A_0\phi -2m_0 c^2 \phi
\]\\
\begin{equation}
  i\hbar \party{\phi}{t} = \frac{\lbr \sigma \cdot\hat{\Pi} \rbr \lbr
    \sigma \cdot\hat{\Pi} \rbr }{2m_0 c} \phi+ eA_0\phi
  \label{nonreldir3}
\end{equation}\\
Ο υπολογισμός του πρώτου όρου του δεύτερου μέλους θα γίνει με τη βοήθεια της σχέσης (\ref{equationforcalc}).  \\
\begin{equation*}
  \begin{split}
    \lbr \sigma \cdot\hat{\Pi} \rbr  \lbr \sigma \cdot\hat{\Pi} \rbr &= \hat{\Pi}^2+i\sigma \cdot \lbr \hat{\Pi} \times \hat{\Pi} \rbr\\
    &=\lbr p-\frac{e}{c}A \rbr^2+i\sigma \cdot \lsbr \lbr p-\frac{e}{c}A \rbr \times \lbr p-\frac{e}{c}A \rbr \rsbr\\
    &=\lbr p-\frac{e}{c}A \rbr^2+i\sigma \cdot \lsbr \lbr i \hbar \nabla-\frac{e}{c}A \rbr \times \lbr i \hbar \nabla-\frac{e}{c}A \rbr \rsbr\\
    &=\lbr p-\frac{e}{c}A \rbr^2+i\sigma \cdot \lsbr \lbr - \hbar^2 \nabla \times \nabla+ \frac{e^2}{c^2}A \times A -i\hbar\frac{e}{c}\nabla \times A -i\hbar \frac{e}{c} A\times \nabla  \rbr \rsbr\\
    &=\lbr p-\frac{e}{c}A \rbr^2-\frac{e}{c}\hbar \sigma \cdot (\nabla \times A)\\
    &=\lbr p-\frac{e}{c}A \rbr^2-\frac{e}{c}\hbar \sigma \cdot B
  \end{split}
\end{equation*}\\
Με βάση τα παραπάνω, η εξίσωση (\ref{nonreldir3}) γίνεται: 
\begin{equation}
  i\hbar \party{\phi}{t} = \lbr \lbr p-\frac{e}{c}A \rbr^2 /2m_0 c - \frac{e}{2 m_0 c}\hbar \sigma \cdot B+ eA_0 \rbr \phi
  \label{nonreldirfinem}
\end{equation}\\
Η (\ref{nonreldirfinem}) δεν είναι άλλη από την εξίσωση \textlatin{Pauli}. Ο χ σπίνορας όπως φάνηκε από την (\ref{nonreldir2}) είναι πολύ μικρός σε σύγκριση με την συνιστώσα φ και γι'αυτο στην προσέγγιση εξετάστηκε το φ. Από το παραπάνω αποτέλεσμα γινεται φανερό ότι η φ  θα περιγράφει τις καταστάσεις σπίν, οπως αναφέρθηκε και σε προηγούμενη παράγραφο. 

\pagebreak


%---------------------
%    bibliografia 
%---------------------

\begin{thebibliography}{9}

\bibitem{traxanas}
  Στέφανος Λ.Τραχανάς,
  \emph{Σχετικιστική Κβαντομηχανική},
  Πανεπιστημιακές Εκδόσεις Κρήτης, 1999.
  %---------------------------------------------

\selectlanguage{english}

\bibitem{greiner}
   w. Greiner,
  \emph{Relativistic Quantum Mechanichs Wave equations},
  Springer-Verlag,1994.
%----------------------
\bibitem{sakurai}
   J.J. Sakurai,
  \emph{Advanced Quantum Mechanics},
  Mass., 1967.

\bibitem{griffiths}
  David J. Griffiths,
  \emph{Introduction to Electrodynamics 3rd ed.},
  Prentice Hall, 1999.
  %---------------------------------------------

\end{thebibliography}

\end{document}
