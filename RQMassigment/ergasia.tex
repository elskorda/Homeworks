\documentclass[12pt,oneside,titlepage,a4paper]{article}
\usepackage{epsfig,scrpage2,graphicx,float,hyperref}
\usepackage{caption,geometry,fullpage}
\usepackage[greek,english]{babel}
\usepackage[utf8x]{inputenc}
\usepackage{subcaption}
\usepackage{multirow}
\setcounter{secnumdepth}{3}
\hypersetup{
  colorlinks=true,
  linkcolor=black,
  citecolor=black,
  urlcolor=blue!50!black
}
\captionsetup{
  labelfont=bf,
  font=small,
  format=hang 
}
%\setlength{\parindent}{0em}
%\setlength{\parskip}{0ex plus0.5ex minus0ex}
%\pagestyle{scrheadings}
\bibliographystyle{unsrt}   
%\oddsidemargin =-1cm
%\setlength{\textwidth}{7in}
%\addtolength{\voffset}{-5pt}
\renewcommand{\headfont}{\normalfont}

\cfoot{\pagemark}

\hypersetup{
  colorlinks=true,
  linkcolor=black,
  citecolor=black,
  urlcolor=blue!50!black
}

\newcommand{\rbr}{
  \ensuremath{\right) }
}

\newcommand{\lbr}{
  \ensuremath{\left( }
}

\newcommand{\rsbr}{
  \ensuremath{\right] }
}

\newcommand{\lsbr}{
  \ensuremath{\left[ }
}

\newcommand{\party}[2]{
  \ensuremath{\frac{\partial #1}{\partial #2}}
}
\newcommand{\deriv}[2]{
  \ensuremath{\frac{d #1}{d#2}}
}

\renewenvironment{abstract}[1][1.0]
{
	\begin{center}
		{\bf Περίληψη}\\[12pt]
		\begin{minipage}{#1\textwidth}
}
{
		\end{minipage}
	\end{center}
}

%------------------------------------------
%begin document 
%------------------------------------------

\begin{document}
%titlepage
\selectlanguage{greek}

\begin{titlepage}
	\begin{figure}[H]
		\centering
		\includegraphics[width = 4cm, keepaspectratio=1]{aristotleUniversityLogo.png}
	\end{figure}
	
	\begin{center}
		\large{{\sc Α.Π.Θ} ΣΧΟΛΗ ΘΕΤΙΚΩΝ ΕΠΙΣΤΗΜΩΝ ΤΜ. ΦΥΣΙΚΗΣ}\\[0.5cm]
		\LARGE\textbf{Εργασία για το μάθημα Σχετικιστική Κβαντομηχανική}\\[1.0cm] 

		\vspace{0.0cm}

		\small{Σκορδά Ελένη}\\[0.2cm]
               
		\small{Διδάσκων καθηγητής: Πασχάλης Ιωάννης }\\[0.2cm]

	\end{center}

	\begin{abstract}
	  Η παρούσα εργασία χωρίζεται σε δύο μέρη. Στο πρώτο μερος αποδεικνύεται ότι για μια συνάρτηση που ικανοποιεί την εξίσωση \textlatin{Klein-Gordon} παρουσία διανυσματικού δυναμικού $A^\mu$, μ = 0,1,2,3, η εξίσωση συνέχειας ικανοποιείται από το τετραδιάνυσμα $J^\mu =\frac{i}{2m} \lbr \Psi* (\partial^\mu \Psi)-\Psi(\partial^\mu)* \rbr - \frac{e}{m}A^\mu \Psi* \Psi$ . Στο δευτερο μέλος της εργασίας μελετάται το μη σχετικιστικό όριο της εξίσωσης \textlatin{Dirac} και \textlatin{Klein Gordon}  
          
	\end{abstract}
        
	\vfill

	\centering{\footnotesize Τμήμα Φυσικής, Α.Π.Θ., \today}
\end{titlepage}

\newpage

% the table of contents is only updated when you run "pdflatex" twice 

\tableofcontents
\newpage 
\section{Μέρος Πρώτο }
\section{Μέρος Δεύτερο }
\section{Βιβλιογραφία }
\end{document}
