\documentclass[12pt,oneside,titlepage,a4paper]{article}
\usepackage{epsfig,scrpage2,graphicx,float,hyperref}
\usepackage{caption,geometry,fullpage}
\usepackage[greek,english]{babel}
\usepackage[utf8x]{inputenc}
\usepackage{subcaption}
\usepackage{multirow}
\setcounter{secnumdepth}{3}
\hypersetup{
  colorlinks=true,
  linkcolor=black,
  citecolor=black,
  urlcolor=blue!50!black
}
\captionsetup{
  labelfont=bf,
  font=small,
  format=hang 
}
%\setlength{\parindent}{0em}
%\setlength{\parskip}{0ex plus0.5ex minus0ex}
%\pagestyle{scrheadings}
\bibliographystyle{unsrt}   
%\oddsidemargin =-1cm
%\setlength{\textwidth}{7in}
%\addtolength{\voffset}{-5pt}
\renewcommand{\headfont}{\normalfont}

\cfoot{\pagemark}

\hypersetup{
  colorlinks=true,
  linkcolor=black,
  citecolor=black,
  urlcolor=blue!50!black
}

\newcommand{\rbr}{
  \ensuremath{\right) }
}

\newcommand{\lbr}{
  \ensuremath{\left( }
}

\newcommand{\rsbr}{
  \ensuremath{\right] }
}

\newcommand{\lsbr}{
  \ensuremath{\left[ }
}

\newcommand{\party}[2]{
  \ensuremath{\frac{\partial #1}{\partial #2}}
}
\newcommand{\deriv}[2]{
  \ensuremath{\frac{d #1}{d#2}}
}

\renewenvironment{abstract}[1][1.0]
{
	\begin{center}
		{\bf Περίληψη}\\[12pt]
		\begin{minipage}{#1\textwidth}
}
{
		\end{minipage}
	\end{center}
}

%------------------------------------------
%begin document 
%------------------------------------------

\begin{document}
%titlepage
\selectlanguage{greek}

\begin{titlepage}
	\begin{figure}[H]
		\centering
		\includegraphics[width = 4cm, keepaspectratio=1]{aristotleUniversityLogo.png}
	\end{figure}
	
	\begin{center}
		\large{{\sc Α.Π.Θ} ΣΧΟΛΗ ΘΕΤΙΚΩΝ ΕΠΙΣΤΗΜΩΝ ΤΜ. ΦΥΣΙΚΗΣ}\\[0.5cm]
		\LARGE\textbf{Εργασία για το μάθημα Σχετικιστική Κβαντομηχανική}\\[1.0cm] 

		\vspace{0.0cm}

		\small{Σκορδά Ελένη}\\[0.2cm]
               
		\small{Διδάσκων καθηγητής: Πασχάλης Ιωάννης }\\[0.2cm]

	\end{center}

	\begin{abstract}
	  Η παρούσα εργασία χωρίζεται σε δύο μέρη. Στο πρώτο μερος αποδεικνύεται ότι για μια συνάρτηση που ικανοποιεί την εξίσωση \textlatin{Klein-Gordon} παρουσία διανυσματικού δυναμικού $A^\mu$, μ = 0,1,2,3, η εξίσωση συνέχειας ικανοποιείται από το τετραδιάνυσμα $J^\mu =\frac{i}{2m} \lbr \Psi* (\partial^\mu \Psi)-\Psi(\partial^\mu)* \rbr - \frac{e}{m}A^\mu \Psi* \Psi$ . Στο δευτερο μέλος της εργασίας μελετάται το μη σχετικιστικό όριο της εξίσωσης \textlatin{Dirac} και \textlatin{Klein Gordon}  
          
	\end{abstract}
        
	\vfill

	\centering{\footnotesize Τμήμα Φυσικής, Α.Π.Θ., \today}
\end{titlepage}

\newpage

% the table of contents is only updated when you run "pdflatex" twice 

\tableofcontents
\newpage


%-----------------------------------------
%eisagwgh
\section{Εισαγωγή}

Η εξίσωση \textlatin{Schr\"ondiger} που χρησιμοποιείται για  την περιγραφή των κβαντομηχανικών σωματιδίων προκύπτει με αντικατάσταση των τελεστών (\ref{energeiaormi}) στην μη σχετικιστική σχέση ενέργειας-ορμής $E^2=\frac{p^2}{2m}$  

\begin{align} 
  E & \rightarrow i\hbar \party{}{t}  &\vec{p} \rightarrow -i \hbar \nabla
  \label{energeiaormi}
\end{align} 

Οι λύσεις της εξίσωσης \textlatin{Schr\"ondiger}, είναι στην περίπτωση του ελεύθερου σωματιδίου, επίπεδα κύματα.

\[
  \Psi(\vec(r),t)=C e^{i(\vec(p)\vec(r)-Et)/\hbar}
\]

Όπως, είναι γνωστό, όμως, η σχέση αυτή δεν μπορεί να περιγράψει σωματίδια με μεγάλες ταχύτητες πολύ κοντά στην ταχύτητα του φωτός. Τα σωματίδια που μελετά η κβαντομηχανική στις περισσότερες περιπτώσεις μπορούν να φτάσουν αυτές τις ταχύτητες, οπότε δημιουργείται η ανάγκη να ληφθεί υπόψη η σχέση ενέργειας-ορμής, της ειδικής θεωρίας της σχετικότητας 

\begin{equation}
  E^2=c^2p^2+m^2c^4
  \label{genrel}
\end{equation} 

Έτσι καταλήγουμε στην εξίσωση \textlatin{Klein-Gordon} ή οποία περιγράφει την κίνηση ενός σωματιδίου το οποίο δεν έχει σπιν. Πέρα από το ότι αδυνατεί να περιγράψει σωματίδια με σπιν, η εξίσωση \textlatin{Klein-Gordon} παρουσιάζει προβλήματα στην ερμηνεία της , όπως θα εξηγηθεί παρακάτω. Για τους παραπάνω λόγους ο \textlatin{Dirac} ακολούθησε διαφορετική μεθοδολογία και κατέληξε στην ομώνυμη εξίσωση, ή οποία περιλαμβάνει και το σπιν. 


Στα κεφάλαια που ακολουθούν θα δοθούν, αρχικά, οι ορισμοί βασικών εννοιών απαραίτητων για την επίλυση των ζητούμενων προβλημάτων.
Κατόπιν, αφού παρουσιαστεί η μορφή της εξίσωσης \textlatin{Klein-Gordon} για ελεύθερο σωματίδιο και σε περίπτωση παρουσίας διανυσματικού δυναμικού θα δειχτεί ότι για μια συνάρτηση που ικανοποιεί την εξίσωση \textlatin{Klein-Gordon} παρουσία διανυσματικού δυναμικού $A^\mu$, μ = 0,1,2,3, η εξίσωση συνέχειας ικανοποιείται από το τετραδιάνυσμα $J^\mu =\frac{i}{2m} \lbr \Psi^* (\partial^\mu \Psi)-\Psi(\partial^\mu)\Psi^* \rbr - \frac{e}{m}A^\mu \Psi^* \Psi$.
Κατόπιν θα γίνει περιγραφή της εξίσωσης \textlatin{Dirac} και θα μελετηθεί το μη σχετικιστικό όριο αυτής αλλά και της \textlatin{Klein Gordon}


\pagebreak


%---------------------
%    bibliografia 
%---------------------

\begin{thebibliography}{9}

\bibitem{traxanas}
  Στέφανος Λ.Τραχανάς,
  \emph{Σχετικιστική Κβαντομηχανική},
  Πανεπιστημιακές Εκδόσεις Κρήτης, 1999.
  %---------------------------------------------
\selectlanguage{english}

\bibitem{greiner}
   w. Greiner,
  \emph{Relativistic Quantum Mechanichs Wave equations},
  Springer-Verlag,1994.
%----------------------
\bibitem{sakurai}
   J.J. Sakurai,
  \emph{Advanced Quantum Mechanics},
  Mass., 1967.
\end{thebibliography}

\end{document}
