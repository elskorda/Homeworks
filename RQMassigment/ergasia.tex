\documentclass[12pt,oneside,titlepage,a4paper]{article}
\usepackage{epsfig,scrpage2,graphicx,float,hyperref}
\usepackage{caption,fullpage,amsmath,amssymb}
\usepackage[greek,english]{babel}
\usepackage[utf8x]{inputenc}
\usepackage{subcaption}
\usepackage{multirow}
\setcounter{secnumdepth}{3}
\hypersetup{
  colorlinks=true,
  linkcolor=black,
  citecolor=black,
  urlcolor=blue!50!black
}
\captionsetup{
  labelfont=bf,
  font=small,
  format=hang 
}
%\setlength{\parindent}{0em}
%\setlength{\parskip}{0ex plus0.5ex minus0ex}
%\pagestyle{scrheadings}
\bibliographystyle{unsrt}   
%\oddsidemargin =-1cm
%\setlength{\textwidth}{7in}
%\addtolength{\voffset}{-5pt}
\renewcommand{\headfont}{\normalfont}

\cfoot{\pagemark}

\hypersetup{
  colorlinks=true,
  linkcolor=black,
  citecolor=black,
  urlcolor=blue!50!black
}

\newcommand{\rbr}{
  \ensuremath{\right) }
}

\newcommand{\lbr}{
  \ensuremath{\left( }
}

\newcommand{\rsbr}{
  \ensuremath{\right] }
}

\newcommand{\lsbr}{
  \ensuremath{\left[ }
}

\newcommand{\party}[2]{
  \ensuremath{\frac{\partial #1}{\partial #2}}
}
\newcommand{\deriv}[2]{
  \ensuremath{\frac{d #1}{d#2}}
}

\renewenvironment{abstract}[1][1.0]
{
	\begin{center}
		{\bf Περίληψη}\\[12pt]
		\begin{minipage}{#1\textwidth}
}
{
		\end{minipage}
	\end{center}
}

%------------------------------------------
%begin document 
%------------------------------------------

\begin{document}
%titlepage
\selectlanguage{greek}

\begin{titlepage}
	\begin{figure}[H]
		\centering
		\includegraphics[width = 4cm, keepaspectratio=1]{aristotleUniversityLogo.png}
	\end{figure}
	
	\begin{center}
		\large{{\sc Α.Π.Θ} ΣΧΟΛΗ ΘΕΤΙΚΩΝ ΕΠΙΣΤΗΜΩΝ ΤΜ. ΦΥΣΙΚΗΣ}\\[0.5cm]
		\LARGE\textbf{Εργασία για το μάθημα Σχετικιστική Κβαντομηχανική}\\[1.0cm] 

		\vspace{0.0cm}

		\small{Σκορδά Ελένη}\\[0.2cm]
               
		\small{Διδάσκων καθηγητής: Πασχάλης Ιωάννης }\\[0.2cm]

	\end{center}

	\begin{abstract}
	  Η παρούσα εργασία χωρίζεται σε δύο μέρη. Στο πρώτο μερος αποδεικνύεται ότι για μια συνάρτηση που ικανοποιεί την εξίσωση \textlatin{Klein-Gordon} παρουσία διανυσματικού δυναμικού $A^\mu$, μ = 0,1,2,3, η εξίσωση συνέχειας ικανοποιείται από το τετραδιάνυσμα $J^\mu =\frac{i}{2m} \lbr \Psi* (\partial^\mu \Psi)-\Psi(\partial^\mu)* \rbr - \frac{e}{m}A^\mu \Psi* \Psi$ . Στο δευτερο μέλος της εργασίας μελετάται το μη σχετικιστικό όριο της εξίσωσης \textlatin{Dirac} και \textlatin{Klein Gordon}  
          
	\end{abstract}
        
	\vfill

	\centering{\footnotesize Τμήμα Φυσικής, Α.Π.Θ., \today}
\end{titlepage}

\newpage

% the table of contents is only updated when you run "pdflatex" twice 

\tableofcontents
\newpage


%-----------------------------------------
%eisagwgh
\section{Εισαγωγή}

Η εξίσωση \textlatin{Schr\"ondiger} που χρησιμοποιείται για  την περιγραφή των κβαντομηχανικών σωματιδίων προκύπτει με αντικατάσταση των τελεστών (\ref{energeiaormi}) στην μη σχετικιστική σχέση ενέργειας-ορμής $E^2=\frac{p^2}{2m}$  

\begin{align} 
  E & \rightarrow i\hbar \party{}{t}  &\vec{p} \rightarrow -i \hbar \nabla
  \label{energeiaormi}
\end{align} 

Οι λύσεις της εξίσωσης \textlatin{Schr\"ondiger}, είναι στην περίπτωση του ελεύθερου σωματιδίου, επίπεδα κύματα.

\[
  \Psi(\vec(r),t)=C e^{i(\vec(p)\vec(r)-Et)/\hbar}
\]

Όπως, είναι γνωστό, όμως, η σχέση αυτή δεν μπορεί να περιγράψει σωματίδια με μεγάλες ταχύτητες πολύ κοντά στην ταχύτητα του φωτός. Τα σωματίδια που μελετά η κβαντομηχανική στις περισσότερες περιπτώσεις μπορούν να φτάσουν αυτές τις ταχύτητες, οπότε δημιουργείται η ανάγκη να ληφθεί υπόψη η σχέση ενέργειας-ορμής, της ειδικής θεωρίας της σχετικότητας 

\begin{equation}
  E^2=c^2p^2+m^2c^4
  \label{genrel}
\end{equation} 

Έτσι καταλήγουμε στην εξίσωση \textlatin{Klein-Gordon} ή οποία περιγράφει την κίνηση ενός σωματιδίου το οποίο δεν έχει σπιν. Πέρα από το ότι αδυνατεί να περιγράψει σωματίδια με σπιν, η εξίσωση \textlatin{Klein-Gordon} παρουσιάζει προβλήματα στην ερμηνεία της , όπως θα εξηγηθεί παρακάτω. Για τους παραπάνω λόγους ο \textlatin{Dirac} ακολούθησε διαφορετική μεθοδολογία και κατέληξε στην ομώνυμη εξίσωση, ή οποία περιλαμβάνει και το σπιν. 


Στα κεφάλαια που ακολουθούν θα δοθούν, αρχικά, οι ορισμοί βασικών εννοιών απαραίτητων για την επίλυση των ζητούμενων προβλημάτων.
Κατόπιν, αφού παρουσιαστεί η μορφή της εξίσωσης \textlatin{Klein-Gordon} για ελεύθερο σωματίδιο και σε περίπτωση παρουσίας διανυσματικού δυναμικού θα δειχτεί ότι για μια συνάρτηση που ικανοποιεί την εξίσωση \textlatin{Klein-Gordon} παρουσία διανυσματικού δυναμικού $A^\mu$, μ = 0,1,2,3, η εξίσωση συνέχειας ικανοποιείται από το τετραδιάνυσμα $J^\mu =\frac{i}{2m} \lbr \Psi^* (\partial^\mu \Psi)-\Psi(\partial^\mu)\Psi^* \rbr - \frac{e}{m}A^\mu \Psi^* \Psi$.
Κατόπιν θα γίνει περιγραφή της εξίσωσης \textlatin{Dirac} και θα μελετηθεί το μη σχετικιστικό όριο αυτής αλλά και της \textlatin{Klein Gordon}


\pagebreak
%-----------------------------------------
%xwroxronos Minkowski
\section{Ειδική Σχετικότητα}

Στο παρόν κεφάλαιο θα δοθούν, συνοπτικά, κάποιοι ορισμοί της ειδικής θεωρίας της σχετικότητας που χρησιμοποιούνται σε επόμενα κεφάλαια με έμφαση, όμως, στο  πως μετασχηματίζονται θεμελιώδεις έννοιες του Ηλεκτρομαγνητισμού στα πλαίσια της θεωρίας αυτής. Επίσης θα εξηγηθούν και ορισμένοι συμβολισμοί. 

\subsection{Τανυστές}


Βασικό μαθηματικό εργαλείο αποτελούν οι τανυστές, που είναι μία γενίκευση των βαθμωτών και διανυσματικών μεγεθών. Υπάρχουν τρία είδη τανυστών :

\begin{itemize}
\item οι συναλλοίωτοι: $ A'^{ij} = \party{x'_i }{x_k} \party{x'_j}{x_l} A^{kl} $
\item οι ανταλλοίωτοι :$ C'_{ij} = \party{x_k}{x'_i} \party{x_l}{x'_j}C_{kl}$ 
\item οι μεικτοί τανυστές: $ B'^i_j  =\party{x'_i }{x_k} \party{x_l}{x'_j} B^l_k  $
\end{itemize}

Κάποιες βασικές και χρήσιμες ιδιότητες , που προκύπτουν σαν άμεση συνέπεια από τα παραπάνω είναι ότι :

\begin{itemize}
\item Αν ένα τανυστής μηδενίζεται σε ένα σύστημα συντεταγμένων , μηδενίζεται σε όλα 
\item Αν δύο τανυστές είναι ίσοι σε κάποιο σύστημα συντεταγμένων τότε θα ισούνται σε όλα 
\item Από τις δυο προηγούμενες ιδιότητες, προκύπτει ότι η μορφή μιας τανυστικής εξίσωσης δεν αλλάζει με την αλλαγή συστήματος 
\end{itemize}
	
Οι πράξεις ανάμεσα σε τανυστές είναι: 

\begin{itemize}
\item Άθροισμα : $C^{ab}_c = \alpha A^{ab}_c +\beta B^{ab}_c $ 
\item Εξωτερικό γινόμενο : ${C^{a}_{bcd}}^e = \alpha A^{a}_b +\beta {B_{ab}}^{e} $ 
\item Συστολή :$ {T_{aq}}^{cq} = \sum\limits_{q=1}^n {T_{aq}}^{cq} $, κατα συνεπεια ο τανυστής $ T^a_a = T $ είναι βαθμωτό μέγεθος.  \
\item Εσωτερικό γινόμενο 
\end{itemize}

\subsection{Χωρόχρονος \textlatin{Minkowski}}

Στην ειδική θεωρία της σχετικότητας ο χώρος δεν είναι ο τρισδιάστατος ευκλείδειος χώρος αλλά 
ο τετραδιάστατος χωρόχρονος \textlatin{Minkowski}. Ο μετρικός τανυστής για το χώρο αυτό ορίζεται ως 

\[
  g_{\mu,\nu} =
  \begin{pmatrix}
    g_{0,0}  & \cdots & g_{0,3} \\
    \vdots & \ddots & \vdots \\
    g_{3,0}  & \cdots & g_{3,3}
  \end{pmatrix}
  =
  \begin{pmatrix}
    1 & 0 & 0 & 0 \\
    0 & -1 & 0 & 0 \\
    0 & 0 & -1 & 0 \\
    0 & 0 & 0 &-1
  \end{pmatrix}
\]

Παρακάτω δίνονται τα τετραδιανύσματα θέσης, ορμής, δυναμικού καθώς και η κλίση σε τέσσερις διαστάσεις:
\begin{align*} 
  x &=\{x,y,z,ict\} 
  \\p &=\{p_x,p_y,p_z,i\frac{E}{c}\}
  \\A &=\{A_x,A_y,A_z,iA_0\}
  \\ \nabla &= \{\party{}{x},\party{}{y},\party{}{z},iA_0\}
\end{align*} 
Το διάνυσμα που αντιπροσωπεύει την θέση ενός σωματιδίου γράφεται,με την βοήθεια του μετρικού τανυστή, για τις συναλλοίωτες και ανταλλοίωτες συντεταγμένες αντοίστιχα: 
\[
x_{mu}=g_{\mu \nu} x^\nu= \{ct,-x,-y,-z\}=\{x_0,x_1,x_2,x_3\}
\]
\[
x^{mu}=g^{\mu \nu} x_\nu= \{x^0,x^1,x^2,x^3\}
\]

Επομένως το εσωτερικό γινόμενο δύο διανυσμάτων θέσης θα είναι:
\[
  x \cdot x =x^\mu x_\mu =x^0 x_0 + x^1 x_1 + x^2 x_2 +x^3 x_3 = ct^2 -x^2-y^2-z^2  
\]

Για την τετραορμή εχουμε: $p^\mu =\{E/c,p_x,p_y,p_z\} $ οπότε για το γινόμενο των ορμών δύο σωματιδίων :

\begin{equation}
  p_1 \cdot p_2 = {p_1}^\mu {p_2}_\mu=\frac{E_1}{c} \frac{E_2}{c}-\vec{p_1}\cdot \vec{p_2} 
  \label{genrel2}
\end{equation}

Ενώ για το εσωτερικό γινόμενο θέσης-ορμής : 
\[
  x\cdot p= x^\mu p_\mu =x_\mu p^\mu= Et-\vec{x}\vec{p}
\]

Ο πίνακας που δίνει το  μετασχηματισμό ενός οποιουδήποτε τετραδιανύσματος ανάμεσα σε δύο αδρανειακά συστήματα, (μετασχηματισμοί \textlatin{Lorentz}) είναι : 

\begin{equation}
  \Lambda_\mu ^\nu=
  \begin{pmatrix}
    \Lambda^0 _0  & \cdots & \Lambda^0 _3 \\
    \vdots & \ddots & \vdots \\
    \Lambda^3 _0  & \cdots & \Lambda^3 _3
  \end{pmatrix}
  =
  \begin{pmatrix}
    \gamma & -\gamma \beta& 0 & 0 \\
    -\gamma \beta & \gamma & 0 & 0 \\
    0 & 0 & 1 & 0 \\
    0 & 0 & 0 & 1
  \end{pmatrix} 
\end{equation}

\subsection{Ειδική θεωρία της σχετικότητας και ηλεκτρομαγνητισμός }
Παρακάτω δίνοται οι σχέσεις μετασχηματισμού για το μαγνητικό και ηλεκτρικό πεδίο:

\begin{align*} 
  E'_x &= E_x        & B'_x &= B_x   
  \\E'_y &= \gamma(E_y-u B_z)         & B'_y &= \gamma \lbr B_y+\frac{u}{c^2}E_z \rbr  
  \\E'_z &= \gamma(E_z+u B_y)        & B'_z &= \gamma \lbr B_z+\frac{u}{c^2}B_z \rbr   
\end{align*} 

Για να βρεθεί η σχέση μετασχηματισμού κάποιου μεγέθους (πχ. ταχύτητα, θέση) αρκεί να υπολογιστεί η δράση του παραπάνω πίνακα στο μέγεθος. Ο μετασχηματισμός ενός τανυστικού μεγέθους δευτέρας τάξης θα είναι επομένως:
\[
k'^{\mu \nu}= {\Lambda^{\mu}}_{\lambda} {\Lambda^{\sigma}}_{\nu} k^{\lambda \sigma} 
\]
Εκτελώντας τις πράξεις το αποτέλεσμα είναι παρόμοιο με τους μετασχηματισμούς για το ηλεκτρομαγνητικό πεδίο. Το γεγονός αυτό οδηγεί στον ορισμό ενός τανυστικού μεγέθους του τανυστή πεδίου. Ο τανυστής αυτός,ουσιαστικά ενοποιεί το ηλεκτρικό και το μαγνητικό πεδίο και δίνει την δυνατότητα(όπως θα δειχθεί παρακάτω) να γραφτούν οι εξισώσεις του \textlatin{Maxwell} σε συμπαγή μορφή.

\[
  F_\mu ^\nu=
  \begin{pmatrix}
    F^0 _0  & \cdots & F^0 _3 \\
    \vdots & \ddots & \vdots \\
    F^3 _0  & \cdots & F^3 _3
  \end{pmatrix}
  =
  \begin{pmatrix}
    0                &\frac{E_x}{c} &\frac{E_y}{c} & \frac{E_z}{c} \\
    -\frac{E_x}{c}   & 0            & B_z          & -B_y \\
    -\frac{E_y}{c}   & -B_z         & 0            & B_x \\
    -\frac{E_z}{c}   & B_y          & -B_x         & 0 \\
  \end{pmatrix} 
\]

Η πυκνότητα φορτίου $\rho = \lbr \frac{Q}{u} \rbr$και ρεύματος $(\vec{J}=\rho \vec{u}) $ μετασχηματίζονται σύμφωνα με τις σχέσεις που δίνονται παρακάτω. 

\begin{align}
  \rho &=\rho_0\frac{1}{\sqrt{1-\frac{u^2}{c^2}}}  & \vec{J}& =\rho_0\frac{\vec{u}}{\sqrt{1-\frac{u^2}{c^2}}}
\end{align}

Οι σχέσεις αυτές είναι παρόμοιες με τις σχέσεις μετασηματισμού των συσνιστωσών ενός τετραδιανύσματος του χωρόχρονου Minkowski, το οποίο είναι : $J^{\mu}=(c\rho,J_x,J_y,Jz)$. 
Η εξισωση συνέχειας γίνεται: 
\begin{equation}
  \nabla J = -\party{\rho}{t} \Rightarrow \sum\limits_{i=1}^3 \party{J^i}{x^i}=\frac{1}{c}\party{J^0}{x^0} \Rightarrow \party{J^{\mu}}{x^{\mu}}=0
  \label{contineq}
\end{equation}
Τέλος οι σχέσεις του \textlatin{Maxwell} γράφονται με την βοήθεια των παραπάνω με μορφή τανυστικής εξίσωσης ως εξής: 
\[
  \party{F^{\mu \nu}}{x^{\nu}}=\mu_0 J^{\nu}
\]

Για το δυναμικό ισχύει: 
\[
  F^{\mu \nu}=\party{A^\nu}{x_{\mu}}-\party{A^\mu}{x_{\nu}}
\]

  

\pagebreak
%-----------------------------------------
%eksiswsi KleinGordon
\section{Εξίσωση \textlatin{Klein-Gordon}}
%============================================================================
%                             new subsection
%============================================================================
\subsection{Εξίσωση \textlatin{Klein-Gordon} για ελεύθερο σωματίδιο}
Για τον υπολογισμό της εξίσωσης \textlatin{Klein-Gordon} αρκεί να αντικατασταθούν οι τελεστές ενέργειας και ορμής (\ref{energeiaormi}) στην (\ref{genrel}): 

\begin{align}
 \notag (i\hbar \party{}{t} )^2 &= (-i\hbar \nabla)^2 c^2 +m_0 ^2 c^4
  \\ -\hbar^2 \party{^2}{t^2} &= -\hbar^2 \nabla^2 c^2 +m_0^2 c^4 
  \label{kg1} 
\end{align}

Για $\hbar=c=1$: 

\[
\lbr \party{^2}{t^2}-\nabla^2 \rbr \Psi +m_0^2 \Psi = 0
\]

Ο πρώτος όρος του αριστερού μέρους είναι η γνωστη ντ'αλαμπερσιανή $\Box = (\party{^2}{t^2}-\nabla^2) $. Επομένως η εξίσωση παίρνει την τελική μορφή: 

\[
  \Box \Psi +m_0^2 \Psi = 0  
\]

Η εξίσωση (\ref{kg1}) μπορεί να γραφτεί και με τανυστική μορφή, αν στην(\ref{genrel2})  αντικατασταθεί το τετραδιάνυσμα της ορμής. Ισχύει ότι : 

\[ 
{p}^\mu {p}_\mu=-\hbar^2 \party{}{x^\mu}\party{}{x_\mu}=\frac{1}{c^2} \party{^2}{t^2} -\nabla
\]

Επομένως η εξίσωση \textlatin{Klein-Gordon} για ελεύθερο σωματίδιο και για $\hbar = c = 1 $ γράφεται: 

\begin{equation}
  {p}^\mu {p}_\mu \Psi + m_0^2 \Psi=0 
  \label{kg2}
\end{equation}

\subsubsection{Λύσεις εξίσωσης \textlatin{Klein Gordon}}
Οι λύσεις της εξίσωσης (\ref{kg2}), είναι επίπεδα κύματα που περιγράφονται από τη μορφή:

\begin{equation}
  \Psi = e^{\lbr -\frac{ip_{\mu}x^{\mu}}{\hbar} \rbr } =e^{ \frac{i}{\hbar}\lbr \vec{p}\vec{x} -Et \rbr }
  \label{sol1}
\end{equation}

Με αντικατάσταση στην εξίσωση (\ref{kg2}), προκύπτει το φάσμα ενεργειών για την ελεύθερη εξίσωση:

\[
  {p}^\mu {p}_\mu e^{\lbr -\frac{ip_{\mu}x^{\mu}}{\hbar} \rbr } +m_0^2 e^{\lbr -\frac{ip_{\mu}x^{\mu}}{\hbar} \rbr } = 0 \Rightarrow {p}^\mu {p}_\mu = m_0^2 c^2 \Rightarrow
\]

\[
\frac{E^2}{c^2}- \vec{p}\vec{p}=m_0^2c^2 \Rightarrow E=\pm \sqrt{\vec{p}^2 + m_0^2c^2} 
\]

Παρατηρείται ότι το φάσμα των επιτρεπτών ενεργειών περιλαμβάνει τόσο θετικές όσο και αρνητικές ενέργειες. Οι αρνητικές ενέργειες συνδέονται με την ύπαρξη αντισωματιδίων. Ένα ακόμα πρόβλημα της εξίσωσης, είναι ότι για τον ακριβή προσδιορισμό της λύσης,είναι απαραίτητο να δινεται και η πρώτη παράγωγος της κυματοσυνάτησης ως αρχική συνθήκη.Ένα ακόμη πρόβλημα προκύπτει με την έκφραση της πυκνότητας πιθανότητας, στην επόμενη παράγραφο.
%============================================================================
%                             new subsection
%============================================================================
\subsection{Εξίσωση συνέχειας και τετραδιάνυσμα ρεύματος για ελεύθερο σωματίδιο}  

Για να υπολογιστεί η μορφή του τετραδιάνυσμα του ρεύματος, υπολογίζεται αρχικά το συζυγές της (\ref{kg1}) :

\[
\lbr \hbar^2 \party{^2}{t^2}- \hbar^2 c^2 \nabla^2 \rbr \Psi^* +m_0^2 c^4 \Psi^* = 0
\]

Κατόπιν η παραπάνω σχέση πολλαπλασιάζεται από τα αριστερά με $\Psi$ και η (\ref{kg1}) από τα αριστερά, πάλι, με $\Psi^*$. Η διαφορά των σχέσεων σχέσεων αυτών δίνει: 

\begin{align*}
  \Psi^* \lbr  \hbar^2 \party{^2}{t^2}- \hbar^2 c^2 \nabla^2 \rbr \Psi +\Psi^* m_0^2 \Psi - \Psi \lbr \hbar^2\party{^2}{t^2}- \hbar^2 c^2 \nabla^2 \rbr \Psi^* -\Psi m_0^2 c^4 \Psi^* &= 0 \Rightarrow
  \\ \hbar^2\Psi^* \party{^2}{t^2} \Psi - \hbar^2 c^2\Psi^* \nabla^2\Psi +\Psi^* c^4 m_0^2 \Psi - \Psi \hbar^2\party{^2}{t^2}\Psi^* + \hbar^2 c^2 \Psi \nabla^2 \Psi^* + \Psi c^4  m_0^2 \Psi^* &= 0 \Rightarrow
\end{align*}

\[
 \frac{1}{c^2}\Psi^* \party{^2}{t^2} \Psi - \Psi^* \nabla^2\Psi  - \frac{1}{c^2} \Psi \party{^2}{t^2}\Psi^* + \Psi \nabla^2 \Psi^* = 0 \Rightarrow
\]

\[
 \frac{1}{c^2} \lbr \Psi^* \party{^2}{t^2} \Psi - \Psi \party{^2}{t^2}\Psi^* \rbr = \Psi^* \nabla^2\Psi- \Psi \nabla^2 \Psi^*  \Rightarrow
\]

\[ 
\frac{1}{c^2} \lbr \Psi^* \party{^2}{t^2} \Psi + \party{\Psi^*}{t} \party{\Psi}{t}-\party{\Psi^*}{t} \party{\Psi}{t}- \Psi \party{^2}{t^2}\Psi^* \rbr = \Psi^* \nabla^2\Psi- \Psi \nabla^2 \Psi^*  \Rightarrow
\]

\[
 \frac{1}{c^2} \lsbr \party{}{t} \lbr \Psi^* \party{\Psi}{t} - \Psi\party{\Psi^*}{t} \rbr \rsbr = \Psi^* \nabla^2\Psi- \Psi \nabla^2 \Psi^*  
\]

\[
\frac{1}{c^2} \lsbr \party{}{t} \lbr \Psi^* \party{\Psi}{t} - \Psi\party{\Psi^*}{t} \rbr \rsbr = \Psi^* \nabla^2\Psi- \Psi \nabla^2 \Psi^*+\nabla \Psi^* \nabla \Psi-  \nabla \Psi^* \nabla \Psi  
\]

\begin{equation}
 \frac{1}{c^2} \lsbr \party{}{t} \lbr \Psi^* \party{\Psi}{t} - \Psi\party{\Psi^*}{t} \rbr \rsbr = \nabla \lbr \Psi^* \nabla \Psi- \Psi \nabla \Psi^* \rbr  
 \label{fineqfree}
\end{equation}

Συγκρίνοντας την εξίσωση συνέχειας (\ref{contineq}) με την προηγούμενη προκύπτει: 

\begin{align}
  \notag \rho &= \frac{1}{c^2} \lbr \Psi^* \party{\Psi}{t} - \Psi\party{\Psi^*}{t} \rbr   &\vec{J}= \Psi^* \nabla \Psi- \Psi \nabla \Psi^*   \Rightarrow
  \\  \rho &= \frac{i\hbar}{2mc^2} \lbr \Psi^* \party{\Psi}{t} - \Psi\party{\Psi^*}{t} \rbr   &\vec{J}= \frac{i\hbar}{2m}\Psi^* \nabla \Psi- \Psi \nabla \Psi^*   
  \label{currentfree}
\end{align}

\vspace{0.5cm}

 Kαι τα τα δύο μέλη της (\ref{fineqfree}) πολλαπλασιάστηκαν με $\frac{i \hbar}{2m}$ ώστε το $\rho$ της (\ref{currentfree}) να έχει διαστάσεις πυκνότητας πιθανότητας.
Παρατηρείται ότι η ερμηνεία του $\rho$ της (\ref{currentfree}) ως πυκνότητας πιθανότητας, δημιουργεί προβήματα,καθως μπορεί να είναι είτε θετικό είτε αρνητικό, κατι που θα σήμαινε ότι το σωματίδιο κάποιες στιγμές θα έπαυε να υπάρχει. Ο λόγος γι'αυτο είναι όπως αναφέρθηκε προηγουμένος ο δευτεροτάξιος χαρακτήρας της εξίσωσης ως προς το χρόνο. 
%============================================================================
%                             new subsection
%============================================================================
\subsection{Εξίσωση \textlatin{Klein-Gordon} παρουσία ηλεκτρομαγνητικού πεδίου}

Στην περίπτωση που υπάρχει παρουσία διανυσματικού πεδίου, στους τελεστές ενέργειας και ορμής (\ref{energeiaormi}) εισάγονται τα δυναμικά $\vec{A}$ και $\phi$ έτσι ώστε ο τρόπος εκλογής τους να μην έχει φυσική σημασία. Η διαδικασία αυτή γνωστή και ως ελάσσονα αντικατάσταση: 

\begin{align} 
  E & \rightarrow i\hbar \party{}{t} -q\phi  &\vec{p} \rightarrow -i \hbar \nabla-q\vec{A}
  \label{energeiaormi2}
\end{align} 

Με τανυστική μορφή μπορεί ο παραπάνω μετασχηματισμός γράφεται : 

\begin{align}
  p^\mu & \rightarrow p^\mu-qA^\mu,    &p_\mu \rightarrow p_\mu-qA_\mu
  \label{tensormin}
\end{align}

Η εξίσωση για το ελεύθερο σωματίδιο, σύμφωνα με τα παραπάνω , μετασχηματίζεται παρουσία πεδιου(για $\hbar=c=1$, και γράφεται ως εξής : 

\[
\lbr i \party{}{t}-q\phi \rbr ^2 \Psi = \lbr -i \nabla -q\vec{A} \rbr ^2 +m_0^2 
\label{kgfield} 
\]

Η τανυστική μορφή της παραπάνω εξίσωσης, προκύπτει αν στην (\ref{kg2}) αντικατασταθεί η (\ref{tensormin}): 

\begin{align}
  \notag &\lbr p^\mu-\frac{q}{c}A^\mu \rbr \lbr {p}_\mu-\frac{q}{c}A_\mu \rbr \Psi +m_0^2 c^2 \Psi = 0 \Rightarrow
  \\  &\lsbr g^{\mu\nu} \lbr i \hbar\party{}{x^\nu}-\frac{q}{c}A_\nu \rbr \lbr i \hbar\party{}{x^\mu}-\frac{q}{c}A_\mu \rbr\rsbr \Psi+m_0^2 c^2 \Psi  =0 
  \label{eqn1}
\end{align}

\subsection{Εξίσωση συνέχειας και τετραδιάνυσμα ρεύματος παρουσία διανυσματικου δυναμικού $A^\mu$}  

Για να αποδειχτεί το ζητούμενο του πρώτου μέρους της εργασίας, θα χρησιμοποιηθεί η τανυστική μορφή της εξίσωσης \textlatin{Klein-Gordon} παρουσία μαγνητικού πεδίου. H μελέτη των πυκνοτήτων ρεύματος και φορτίου, θα γίνει με  παρόμοια  διαδικασία  με της παραγράφου 3.2. Συγκεκριμένα, η \ref{eqn1} θα πολλαπλασιαστεί από τα αριστερά με $\Psi^*$ (\ref{eqn2}) 
%-------------------------------------------------
\begin{align}
  \notag &\Psi^* \lsbr g^{\mu\nu} \lbr i \hbar\party{}{x^\nu}-\frac{q}{c}A_\nu \rbr \lbr i \hbar\party{}{x^\mu}-\frac{q}{c}A_\mu \rbr\rsbr \Psi + \Psi^* m_0^2 c^2 \Psi=0 
  \\&\Psi^* \lsbr -\hbar^2 g^{\mu\nu} \lbr \party{}{x^\nu}+\frac{iq}{\hbar c}A_\nu \rbr \lbr \party{}{x^\mu}+\frac{iq}{\hbar c}A_\mu \rbr\rsbr \Psi + \Psi^* m_0^2 c^2 \Psi=0 
  \label{eqn2}
\end{align}
%-------------------------------------------------
Κατόπιν υπολογίζεται το συζυγές της προηγούμενης εξίσωσης (\ref{eqn3})
\begin{align}
  \notag  &\left\{ \Psi^* \lsbr -\hbar^2 g^{\mu\nu} \lbr \party{}{x^\nu}+\frac{iq}{\hbar c}A_\nu \rbr \lbr \party{}{x^\mu}+\frac{iq}{\hbar c}A_\mu \rbr\rsbr \Psi + \Psi^* m_0^2 c^2 \Psi=0  \right\} ^*=0 &\Rightarrow
  \\ &\Psi \lsbr -\hbar^2 g^{\mu\nu} \lbr \party{}{x^\nu}-\frac{iq}{\hbar c}A_\nu \rbr \lbr \party{}{x^\mu}-\frac{iq}{\hbar c}A_\mu \rbr\rsbr \Psi^* + \Psi m_0^2 c^2 \Psi^*=0 
  \label{eqn3}
\end{align}
Τέλος λαμβάνεται η διαφορά των εξισώσεων(\ref{eqn2}) και (\ref{eqn3}) που θα οδηγήσει όπως και στην προηγούμενη παράγραφο σε μια εξίσωση, που η σύγκριση της με την εξίσωση συνέχειας, θα δωσει το ζητούμενο του πρώτου μέρους, την μορφή του τετραδιανύσματος του ρεύματος. 


\begin{align*}
  &\Psi^* \lsbr -\hbar^2 g^{\mu\nu} \lbr \party{}{x^\nu}+\frac{iq}{\hbar c}A_\nu \rbr \lbr \party{}{x^\mu}+\frac{iq}{\hbar c}A_\mu \rbr\rsbr \Psi- 
  \\ &-\Psi \lsbr -\hbar^2 g^{\mu\nu} \lbr \party{}{x^\nu}-\frac{iq}{\hbar c}A_\nu \rbr \lbr \party{}{x^\mu}-\frac{iq}{\hbar c}A_\mu \rbr\rsbr \Psi^* &=0 &\Rightarrow
  \\
  \\ &\Psi \lsbr g^{\mu\nu} \lbr \party{}{x^\nu}-\frac{iq}{\hbar c}A_\nu \rbr \lbr \party{}{x^\mu}-\frac{iq}{\hbar c}A_\mu \rbr\rsbr \Psi^*- 
  \\ & -\Psi^* \lsbr g^{\mu\nu} \lbr \party{}{x^\nu}+\frac{iq}{\hbar c}A_\nu \rbr \lbr \party{}{x^\mu}+\frac{iq}{\hbar c}A_\mu \rbr\rsbr \Psi &=0 &\Rightarrow
\end{align*}

\begin{align*}
  &g^{\mu\nu} \Psi \party{}{x^\nu} \party{}{x^\mu} \Psi^*-   g^{\mu\nu} \Psi \frac{iq}{\hbar c}\party{}{x^\nu}(A_\mu\Psi^*)-g^{\mu\nu}  \Psi \frac{iq}{\hbar c} A_\nu \party{}{x^\mu} \Psi^*  -g^{\mu\nu} \Psi \frac{q^2}{c^2}A_\nu A_\mu \Psi^* -
  \\& -g^{\mu\nu} \Psi^* \party{}{x^\nu} \party{}{x^\mu} \Psi - g^{\mu\nu} \Psi^* \frac{iq}{\hbar c}\party{}{x^\nu}(A_\mu \Psi) - g^{\mu\nu}  \Psi^*  \frac{iq}{\hbar c} A_\nu \party{}{x^\mu} \Psi  +g^{\mu\nu} \Psi \frac{q^2}{c^2}A_\nu A_\mu \Psi^* =0  &\Rightarrow
  %-----------------------
  \\
  \\ &g^{\mu\nu} \Psi \party{}{x^\nu} \party{}{x^\mu} \Psi^*-   g^{\mu\nu} \Psi \frac{iq}{\hbar c}\party{}{x^\nu}(A_\mu\Psi^*)-g^{\mu\nu}  \Psi \frac{iq}{\hbar c} A_\nu \party{}{x^\mu} \Psi^*-g^{\mu\nu} \Psi^* \party{}{x^\nu} \party{}{x^\mu} \Psi -
  \\&-g^{\mu\nu} \Psi^* \frac{iq}{\hbar c}\party{}{x^\nu}(A_\mu \Psi) - g^{\mu\nu}  \Psi^*  \frac{iq}{\hbar c} A_\nu \party{}{x^\mu} \Psi =0  &\Rightarrow 
  %------------------------  
  \\
  \\ &\lbr g^{\mu\nu} \Psi \party{}{x^\nu} \party{}{x^\mu} \Psi^*-g^{\mu\nu} \Psi^* \party{}{x^\nu} \party{}{x^\mu} \Psi+ \party{\Psi^*}{x^\nu} \party{\Psi}{x^\mu}-\party{\Psi}{x^\nu} \party{\Psi^*}{x^\mu} \rbr -   g^{\mu\nu} \Psi \frac{iq}{\hbar c}\party{}{x^\nu}(A_\mu\Psi^*)-
  \\&-g^{\mu\nu}  \Psi \frac{iq}{\hbar c} A_\nu \party{}{x^\mu} \Psi^* -g^{\mu\nu} \Psi^* \frac{iq}{\hbar c}\party{}{x^\nu}(A_\mu \Psi) - g^{\mu\nu}  \Psi^*  \frac{iq}{\hbar c} A_\nu \party{}{x^\mu} \Psi =0  &\Rightarrow 
  %------------------------  
  \\
  \\ &g^{\mu\nu} \party{}{x^\mu} \lbr   \Psi\party{}{x^\nu}\Psi^*  - \Psi^* \party{}{x^\nu} \Psi \rbr - g^{\mu\nu} \Psi \frac{iq}{\hbar c}\party{}{x^\nu}(A_\mu\Psi^*)-g^{\mu\nu}  \Psi \frac{iq}{\hbar c} A_\nu \party{}{x^\mu} \Psi^*-
  \\& -g^{\mu\nu} \Psi^* \frac{iq}{\hbar c}\party{}{x^\nu}(A_\mu \Psi) - g^{\mu\nu}  \Psi^*  \frac{iq}{\hbar c} A_\nu \party{}{x^\mu} \Psi =0  &\Rightarrow 
  %------------------------  
  \\
  \\ &g^{\mu\nu} \party{}{x^\mu} \lbr   \Psi\party{}{x^\nu}\Psi^*  - \Psi^* \party{}{x^\nu} \Psi \rbr - g^{\mu\nu} \Psi \frac{iq}{\hbar c} A_\mu\party{\Psi^*}{x^\nu}- g^{\mu\nu} \Psi \Psi^* \frac{iq}{\hbar c}\party{A_\mu}{x^\nu}-
 \\&-g^{\mu\nu}\Psi \frac{iq}{\hbar c} A_\nu \party{\Psi^*}{x^\mu} -g^{\mu\nu} \Psi^*  \Psi\frac{iq}{\hbar c}\party{A_\mu}{x^\nu}-g^{\mu\nu} \Psi^* \frac{iq}{\hbar c}A_\mu\party{\Psi}{x^\nu}- 
  \\&-g^{\mu\nu}  \Psi^*  \frac{iq}{\hbar c} A_\nu \party{}{x^\mu} \Psi =0  &\Rightarrow
  %------------------------  
  \\
  \\ &g^{\mu\nu} \party{}{x^\mu} \lbr   \Psi\party{}{x^\nu}\Psi^*  - \Psi^* \party{}{x^\nu} \Psi \rbr - g^{\mu\nu} \frac{iq}{\hbar c} \lbr 2\Psi A_\mu\party{\Psi^*}{x^\mu} +2\Psi \Psi^* \party{A_\mu}{x^\mu}+ 2 A_\mu \Psi^* \party{\Psi}{x^\mu} \rbr=0 &\Rightarrow
\end{align*}

\begin{align}
  \notag &g^{\mu\nu} \party{}{x^\mu} \lbr   \Psi\party{}{x^\nu}\Psi^*  - \Psi^* \party{}{x^\nu} \Psi \rbr - g^{\mu\nu} \frac{iq}{\hbar c} \party{}{x^\mu}\lbr 2\Psi A_\mu \Psi^* \rbr=0 &\Rightarrow
  \notag\\
  \\& g^{\mu\nu} \party{}{x^\mu} \lsbr  \lbr \Psi\party{}{x^\nu}\Psi^*  - \Psi^* \party{}{x^\nu} \Psi \rbr - \frac{2iq}{\hbar c} \lbr \Psi A_\mu \Psi^* \rbr \rsbr = 0
  \label{befin}
\end{align}
\vspace{0.5cm}

Λόγω των εξισώσεων (\ref{dianisma}) και της (\ref{contineq2}) η (\ref{befin}) γίνεται : 
\vspace{0.5cm}
\begin{mdframed}
  \begin{equation}
    J^\mu =\frac{i \hbar}{2m} \lbr \Psi^* (\partial ^\mu \Psi)-\Psi(\partial ^\mu \Psi^*) \rbr - \frac{iq}{mc} A_\mu \Psi^* \Psi 
    \label{finqchar}
  \end{equation}
\end{mdframed}
\vspace{0.5cm}
Kαι τα τα δύο μέλη της (\ref{finqchar}) πολλαπλασιάστηκαν με $\frac{i \hbar}{2m}$. Δείχτηκε επομένως ότι για μια συνάρτηση που ικανοποιεί την εξίσωση \textlatin{Klein-Gordon} παρουσία διανυσματικού δυναμικού $A_\mu$, μ = 0,1,2,3, η εξίσωση συνέχειας ικανοποιείται από το τετραδιάνυσμα της σχέσης (\ref{finqchar}). 



\pagebreak


%---------------------
%    bibliografia 
%---------------------

\begin{thebibliography}{9}

\bibitem{traxanas}
  Στέφανος Λ.Τραχανάς,
  \emph{Σχετικιστική Κβαντομηχανική},
  Πανεπιστημιακές Εκδόσεις Κρήτης, 1999.
  %---------------------------------------------

\selectlanguage{english}

\bibitem{greiner}
   w. Greiner,
  \emph{Relativistic Quantum Mechanichs Wave equations},
  Springer-Verlag,1994.
%----------------------
\bibitem{sakurai}
   J.J. Sakurai,
  \emph{Advanced Quantum Mechanics},
  Mass., 1967.

\bibitem{griffiths}
  David J. Griffiths,
  \emph{Introduction to Electrodynamics 3rd ed.},
  Prentice Hall, 1999.
  %---------------------------------------------

\end{thebibliography}

\end{document}
