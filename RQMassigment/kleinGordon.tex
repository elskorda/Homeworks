\section{Εξίσωση \textlatin{Klein-Gordon}}
%============================================================================
%                             new subsection
%============================================================================
\subsection{Εξίσωση \textlatin{Klein-Gordon} για ελεύθερο σωματίδιο}
Για τον υπολογισμό της εξίσωσης \textlatin{Klein-Gordon} αρκεί να αντικατασταθούν οι τελεστές ενέργειας και ορμής (\ref{energeiaormi}) στην (\ref{genrel}): 

\begin{align}
 \notag (i\hbar \party{}{t} )^2 &= (-i\hbar \nabla)^2 c^2 +m_0 ^2 c^4
  \\ -\hbar^2 \party{^2}{t^2} &= -\hbar^2 \nabla^2 c^2 +m_0^2 c^4 
  \label{kg1} 
\end{align}

Για $\hbar=c=1$: 

\[
\lbr \party{^2}{t^2}-\nabla^2 \rbr \Psi +m_0^2 \Psi = 0
\]

Ο πρώτος όρος του αριστερού μέρους είναι η γνωστη ντ'αλαμπερσιανή $\Box = (\party{^2}{t^2}-\nabla^2) $. Επομένως η εξίσωση παίρνει την τελική μορφή: 

\[
  \Box \Psi +m_0^2 \Psi = 0  
\]

Η εξίσωση (\ref{kg1}) μπορεί να γραφτεί και με τανυστική μορφή, αν στην(\ref{genrel2})  αντικατασταθεί το τετραδιάνυσμα της ορμής. Ισχύει ότι : 

\[ 
{p}^\mu {p}_\mu=-\hbar^2 \party{}{x^\mu}\party{}{x_\mu}=\frac{1}{c^2} \party{^2}{t^2} -\nabla
\]

Επομένως η εξίσωση \textlatin{Klein-Gordon} για ελεύθερο σωματίδιο και για $\hbar = c = 1 $ γράφεται: 

\begin{equation}
  {p}^\mu {p}_\mu \Psi + m_0^2 \Psi=0 
  \label{kg2}
\end{equation}

\subsubsection{Λύσεις εξίσωσης \textlatin{Klein Gordon}}
Οι λύσεις της εξίσωσης (\ref{kg2}), είναι επίπεδα κύματα που περιγράφονται από τη μορφή:

\begin{equation}
  \Psi = e^{\lbr -\frac{ip_{\mu}x^{\mu}}{\hbar} \rbr } =e^{ \frac{i}{\hbar}\lbr \vec{p}\vec{x} -Et \rbr }
  \label{sol1}
\end{equation}

Με αντικατάσταση στην εξίσωση (\ref{kg2}), προκύπτει το φάσμα ενεργειών για την ελεύθερη εξίσωση:

\[
  {p}^\mu {p}_\mu e^{\lbr -\frac{ip_{\mu}x^{\mu}}{\hbar} \rbr } +m_0^2 e^{\lbr -\frac{ip_{\mu}x^{\mu}}{\hbar} \rbr } = 0 \Rightarrow {p}^\mu {p}_\mu = m_0^2 c^2 \Rightarrow
\]

\[
\frac{E^2}{c^2}- \vec{p}\vec{p}=m_0^2c^2 \Rightarrow E=\pm \sqrt{\vec{p}^2 + m_0^2c^2} 
\]

Παρατηρείται ότι το φάσμα των επιτρεπτών ενεργειών περιλαμβάνει τόσο θετικές όσο και αρνητικές ενέργειες. Οι αρνητικές ενέργειες συνδέονται με την ύπαρξη αντισωματιδίων. Ένα ακόμα πρόβλημα της εξίσωσης, είναι ότι για τον ακριβή προσδιορισμό της λύσης,είναι απαραίτητο να δινεται και η πρώτη παράγωγος της κυματοσυνάτησης ως αρχική συνθήκη.Παρατηρείται ότι πουθενά στην εξίσωση δεν υπεισέρχεται η έννοια του σπίν. Αυτό σημαίνει ότι η εξίσωση αυτή μπορεί να εφαρμοστεί μόνο για σωματίδια με σπιν μηδέν.Ένα ακόμη πρόβλημα προκύπτει με την έκφραση της πυκνότητας πιθανότητας, στην επόμενη παράγραφο.
%============================================================================
%                             new subsection
%============================================================================
\subsection{Εξίσωση συνέχειας και τετραδιάνυσμα ρεύματος για ελεύθερο σωματίδιο}  

Για να υπολογιστεί η μορφή του τετραδιάνυσμα του ρεύματος, υπολογίζεται αρχικά το συζυγές της (\ref{kg1}) :

\[
\lbr \hbar^2 \party{^2}{t^2}- \hbar^2 c^2 \nabla^2 \rbr \Psi^* +m_0^2 c^4 \Psi^* = 0
\]

Κατόπιν η παραπάνω σχέση πολλαπλασιάζεται από τα αριστερά με $\Psi$ και η (\ref{kg1}) από τα αριστερά, πάλι, με $\Psi^*$. Η διαφορά των σχέσεων σχέσεων αυτών δίνει: 

\begin{align*}
  \Psi^* \lbr  \hbar^2 \party{^2}{t^2}- \hbar^2 c^2 \nabla^2 \rbr \Psi +\Psi^* m_0^2 \Psi - \Psi \lbr \hbar^2\party{^2}{t^2}- \hbar^2 c^2 \nabla^2 \rbr \Psi^* -\Psi m_0^2 c^4 \Psi^* &= 0 \Rightarrow
  \\ \hbar^2\Psi^* \party{^2}{t^2} \Psi - \hbar^2 c^2\Psi^* \nabla^2\Psi +\Psi^* c^4 m_0^2 \Psi - \Psi \hbar^2\party{^2}{t^2}\Psi^* + \hbar^2 c^2 \Psi \nabla^2 \Psi^* + \Psi c^4  m_0^2 \Psi^* &= 0 \Rightarrow
\end{align*}

\[
 \frac{1}{c^2}\Psi^* \party{^2}{t^2} \Psi - \Psi^* \nabla^2\Psi  - \frac{1}{c^2} \Psi \party{^2}{t^2}\Psi^* + \Psi \nabla^2 \Psi^* = 0 \Rightarrow
\]

\[
 \frac{1}{c^2} \lbr \Psi^* \party{^2}{t^2} \Psi - \Psi \party{^2}{t^2}\Psi^* \rbr = \Psi^* \nabla^2\Psi- \Psi \nabla^2 \Psi^*  \Rightarrow
\]

\[ 
\frac{1}{c^2} \lbr \Psi^* \party{^2}{t^2} \Psi + \party{\Psi^*}{t} \party{\Psi}{t}-\party{\Psi^*}{t} \party{\Psi}{t}- \Psi \party{^2}{t^2}\Psi^* \rbr = \Psi^* \nabla^2\Psi- \Psi \nabla^2 \Psi^*  \Rightarrow
\]

\[
 \frac{1}{c^2} \lsbr \party{}{t} \lbr \Psi^* \party{\Psi}{t} - \Psi\party{\Psi^*}{t} \rbr \rsbr = \Psi^* \nabla^2\Psi- \Psi \nabla^2 \Psi^*  
\]

\[
\frac{1}{c^2} \lsbr \party{}{t} \lbr \Psi^* \party{\Psi}{t} - \Psi\party{\Psi^*}{t} \rbr \rsbr = \Psi^* \nabla^2\Psi- \Psi \nabla^2 \Psi^*+\nabla \Psi^* \nabla \Psi-  \nabla \Psi^* \nabla \Psi  
\]

\begin{equation}
 \frac{1}{c^2} \lsbr \party{}{t} \lbr \Psi^* \party{\Psi}{t} - \Psi\party{\Psi^*}{t} \rbr \rsbr = \nabla \lbr \Psi^* \nabla \Psi- \Psi \nabla \Psi^* \rbr  
 \label{fineqfree}
\end{equation}

Συγκρίνοντας την εξίσωση συνέχειας (\ref{contineq}) με την προηγούμενη προκύπτει: 

\begin{align}
  \notag \rho &= \frac{1}{c^2} \lbr \Psi^* \party{\Psi}{t} - \Psi\party{\Psi^*}{t} \rbr   &\vec{J}= \Psi^* \nabla \Psi- \Psi \nabla \Psi^*   \Rightarrow
  \\  \rho &= \frac{i\hbar}{2mc^2} \lbr \Psi^* \party{\Psi}{t} - \Psi\party{\Psi^*}{t} \rbr   &\vec{J}= \frac{i\hbar}{2m}\Psi^* \nabla \Psi- \Psi \nabla \Psi^*   
  \label{currentfree}
\end{align}

\vspace{0.5cm}

 Kαι τα τα δύο μέλη της (\ref{fineqfree}) πολλαπλασιάστηκαν με $\frac{i \hbar}{2m}$ ώστε το $\rho$ της (\ref{currentfree}) να έχει διαστάσεις πυκνότητας πιθανότητας.
Παρατηρείται ότι η ερμηνεία του $\rho$ της (\ref{currentfree}) ως πυκνότητας πιθανότητας, δημιουργεί προβήματα,καθως μπορεί να είναι είτε θετικό είτε αρνητικό, κατι που θα σήμαινε ότι το σωματίδιο κάποιες στιγμές θα έπαυε να υπάρχει. Ο λόγος γι'αυτο είναι όπως αναφέρθηκε προηγουμένος ο δευτεροτάξιος χαρακτήρας της εξίσωσης ως προς το χρόνο. 
%============================================================================
%                             new subsection
%============================================================================
\subsection{Εξίσωση \textlatin{Klein-Gordon} παρουσία ηλεκτρομαγνητικού πεδίου}

Στην περίπτωση που υπάρχει παρουσία διανυσματικού πεδίου, στους τελεστές ενέργειας και ορμής (\ref{energeiaormi}) εισάγονται τα δυναμικά $\vec{A}$ και $\phi$ έτσι ώστε ο τρόπος εκλογής τους να μην έχει φυσική σημασία. Η διαδικασία αυτή γνωστή και ως ελάσσονα αντικατάσταση: 

\begin{align} 
  E & \rightarrow i\hbar \party{}{t} -q\phi  &\vec{p} \rightarrow -i \hbar \nabla-q\vec{A}
  \label{energeiaormi2}
\end{align} 

Με τανυστική μορφή μπορεί ο παραπάνω μετασχηματισμός γράφεται : 

\begin{align}
  p^\mu & \rightarrow p^\mu-qA^\mu,    &p_\mu \rightarrow p_\mu-qA_\mu
  \label{tensormin}
\end{align}

Η εξίσωση για το ελεύθερο σωματίδιο, σύμφωνα με τα παραπάνω , μετασχηματίζεται παρουσία πεδιου(για $\hbar=c=1$, και γράφεται ως εξής : 

\[
\lbr i \party{}{t}-q\phi \rbr ^2 \Psi = \lbr -i \nabla -q\vec{A} \rbr ^2 +m_0^2 
\label{kgfield} 
\]

Η τανυστική μορφή της παραπάνω εξίσωσης, προκύπτει αν στην (\ref{kg2}) αντικατασταθεί η (\ref{tensormin}): 

\begin{align}
  \notag &\lbr p^\mu-\frac{q}{c}A^\mu \rbr \lbr {p}_\mu-\frac{q}{c}A_\mu \rbr \Psi +m_0^2 c^2 \Psi = 0 \Rightarrow
  \\  &\lsbr g^{\mu\nu} \lbr i \hbar\party{}{x^\nu}-\frac{q}{c}A_\nu \rbr \lbr i \hbar\party{}{x^\mu}-\frac{q}{c}A_\mu \rbr\rsbr \Psi+m_0^2 c^2 \Psi  =0 
  \label{eqn1}
\end{align}

\subsection{Εξίσωση συνέχειας και τετραδιάνυσμα ρεύματος παρουσία διανυσματικου δυναμικού $A^\mu$}  

Για να αποδειχτεί το ζητούμενο του πρώτου μέρους της εργασίας, θα χρησιμοποιηθεί η τανυστική μορφή της εξίσωσης \textlatin{Klein-Gordon} παρουσία μαγνητικού πεδίου. H μελέτη των πυκνοτήτων ρεύματος και φορτίου, θα γίνει με  παρόμοια  διαδικασία  με της παραγράφου 3.2. Συγκεκριμένα, η \ref{eqn1} θα πολλαπλασιαστεί από τα αριστερά με $\Psi^*$ (\ref{eqn2}) 
%-------------------------------------------------
\begin{align}
  \notag &\Psi^* \lsbr g^{\mu\nu} \lbr i \hbar\party{}{x^\nu}-\frac{q}{c}A_\nu \rbr \lbr i \hbar\party{}{x^\mu}-\frac{q}{c}A_\mu \rbr\rsbr \Psi + \Psi^* m_0^2 c^2 \Psi=0 
  \\&\Psi^* \lsbr -\hbar^2 g^{\mu\nu} \lbr \party{}{x^\nu}+\frac{iq}{\hbar c}A_\nu \rbr \lbr \party{}{x^\mu}+\frac{iq}{\hbar c}A_\mu \rbr\rsbr \Psi + \Psi^* m_0^2 c^2 \Psi=0 
  \label{eqn2}
\end{align}
%-------------------------------------------------
Κατόπιν υπολογίζεται το συζυγές της προηγούμενης εξίσωσης (\ref{eqn3})
\begin{align}
  \notag  &\left\{ \Psi^* \lsbr -\hbar^2 g^{\mu\nu} \lbr \party{}{x^\nu}+\frac{iq}{\hbar c}A_\nu \rbr \lbr \party{}{x^\mu}+\frac{iq}{\hbar c}A_\mu \rbr\rsbr \Psi + \Psi^* m_0^2 c^2 \Psi=0  \right\} ^*=0 &\Rightarrow
  \\ &\Psi \lsbr -\hbar^2 g^{\mu\nu} \lbr \party{}{x^\nu}-\frac{iq}{\hbar c}A_\nu \rbr \lbr \party{}{x^\mu}-\frac{iq}{\hbar c}A_\mu \rbr\rsbr \Psi^* + \Psi m_0^2 c^2 \Psi^*=0 
  \label{eqn3}
\end{align}
Τέλος λαμβάνεται η διαφορά των εξισώσεων(\ref{eqn2}) και (\ref{eqn3}) που θα οδηγήσει όπως και στην προηγούμενη παράγραφο σε μια εξίσωση, που η σύγκριση της με την εξίσωση συνέχειας, θα δωσει το ζητούμενο του πρώτου μέρους, την μορφή του τετραδιανύσματος του ρεύματος. 


\begin{align*}
  &\Psi^* \lsbr -\hbar^2 g^{\mu\nu} \lbr \party{}{x^\nu}+\frac{iq}{\hbar c}A_\nu \rbr \lbr \party{}{x^\mu}+\frac{iq}{\hbar c}A_\mu \rbr\rsbr \Psi- 
  \\ &-\Psi \lsbr -\hbar^2 g^{\mu\nu} \lbr \party{}{x^\nu}-\frac{iq}{\hbar c}A_\nu \rbr \lbr \party{}{x^\mu}-\frac{iq}{\hbar c}A_\mu \rbr\rsbr \Psi^* &=0 &\Rightarrow
  \\
  \\ &\Psi \lsbr g^{\mu\nu} \lbr \party{}{x^\nu}-\frac{iq}{\hbar c}A_\nu \rbr \lbr \party{}{x^\mu}-\frac{iq}{\hbar c}A_\mu \rbr\rsbr \Psi^*- 
  \\ & -\Psi^* \lsbr g^{\mu\nu} \lbr \party{}{x^\nu}+\frac{iq}{\hbar c}A_\nu \rbr \lbr \party{}{x^\mu}+\frac{iq}{\hbar c}A_\mu \rbr\rsbr \Psi &=0 &\Rightarrow
\end{align*}

\begin{align*}
  &g^{\mu\nu} \Psi \party{}{x^\nu} \party{}{x^\mu} \Psi^*-   g^{\mu\nu} \Psi \frac{iq}{\hbar c}\party{}{x^\nu}(A_\mu\Psi^*)-g^{\mu\nu}  \Psi \frac{iq}{\hbar c} A_\nu \party{}{x^\mu} \Psi^*  -g^{\mu\nu} \Psi \frac{q^2}{c^2}A_\nu A_\mu \Psi^* -
  \\& -g^{\mu\nu} \Psi^* \party{}{x^\nu} \party{}{x^\mu} \Psi - g^{\mu\nu} \Psi^* \frac{iq}{\hbar c}\party{}{x^\nu}(A_\mu \Psi) - g^{\mu\nu}  \Psi^*  \frac{iq}{\hbar c} A_\nu \party{}{x^\mu} \Psi  +g^{\mu\nu} \Psi \frac{q^2}{c^2}A_\nu A_\mu \Psi^* =0  &\Rightarrow
  %-----------------------
  \\
  \\ &g^{\mu\nu} \Psi \party{}{x^\nu} \party{}{x^\mu} \Psi^*-   g^{\mu\nu} \Psi \frac{iq}{\hbar c}\party{}{x^\nu}(A_\mu\Psi^*)-g^{\mu\nu}  \Psi \frac{iq}{\hbar c} A_\nu \party{}{x^\mu} \Psi^*-g^{\mu\nu} \Psi^* \party{}{x^\nu} \party{}{x^\mu} \Psi -
  \\&-g^{\mu\nu} \Psi^* \frac{iq}{\hbar c}\party{}{x^\nu}(A_\mu \Psi) - g^{\mu\nu}  \Psi^*  \frac{iq}{\hbar c} A_\nu \party{}{x^\mu} \Psi =0  &\Rightarrow 
  %------------------------  
  \\
  \\ &\lbr g^{\mu\nu} \Psi \party{}{x^\nu} \party{}{x^\mu} \Psi^*-g^{\mu\nu} \Psi^* \party{}{x^\nu} \party{}{x^\mu} \Psi+ \party{\Psi^*}{x^\nu} \party{\Psi}{x^\mu}-\party{\Psi}{x^\nu} \party{\Psi^*}{x^\mu} \rbr -   g^{\mu\nu} \Psi \frac{iq}{\hbar c}\party{}{x^\nu}(A_\mu\Psi^*)-
  \\&-g^{\mu\nu}  \Psi \frac{iq}{\hbar c} A_\nu \party{}{x^\mu} \Psi^* -g^{\mu\nu} \Psi^* \frac{iq}{\hbar c}\party{}{x^\nu}(A_\mu \Psi) - g^{\mu\nu}  \Psi^*  \frac{iq}{\hbar c} A_\nu \party{}{x^\mu} \Psi =0  &\Rightarrow 
\end{align*}
%------------------------  
\begin{align*}
  \\
  \\ &g^{\mu\nu} \party{}{x^\mu} \lbr   \Psi\party{}{x^\nu}\Psi^*  - \Psi^* \party{}{x^\nu} \Psi \rbr - g^{\mu\nu} \Psi \frac{iq}{\hbar c}\party{}{x^\nu}(A_\mu\Psi^*)-g^{\mu\nu}  \Psi \frac{iq}{\hbar c} A_\nu \party{}{x^\mu} \Psi^*-
  \\& -g^{\mu\nu} \Psi^* \frac{iq}{\hbar c}\party{}{x^\nu}(A_\mu \Psi) - g^{\mu\nu}  \Psi^*  \frac{iq}{\hbar c} A_\nu \party{}{x^\mu} \Psi =0  &\Rightarrow 
  %------------------------  
  \\
  \\ &g^{\mu\nu} \party{}{x^\mu} \lbr   \Psi\party{}{x^\nu}\Psi^*  - \Psi^* \party{}{x^\nu} \Psi \rbr - g^{\mu\nu} \Psi \frac{iq}{\hbar c} A_\mu\party{\Psi^*}{x^\nu}- g^{\mu\nu} \Psi \Psi^* \frac{iq}{\hbar c}\party{A_\mu}{x^\nu}-
 \\&-g^{\mu\nu}\Psi \frac{iq}{\hbar c} A_\nu \party{\Psi^*}{x^\mu} -g^{\mu\nu} \Psi^*  \Psi\frac{iq}{\hbar c}\party{A_\mu}{x^\nu}-g^{\mu\nu} \Psi^* \frac{iq}{\hbar c}A_\mu\party{\Psi}{x^\nu}- 
  \\&-g^{\mu\nu}  \Psi^*  \frac{iq}{\hbar c} A_\nu \party{}{x^\mu} \Psi =0  &\Rightarrow
  %------------------------  
  \\
  \\ &g^{\mu\nu} \party{}{x^\mu} \lbr   \Psi\party{}{x^\nu}\Psi^*  - \Psi^* \party{}{x^\nu} \Psi \rbr - g^{\mu\nu} \frac{iq}{\hbar c} \lbr 2\Psi A_\mu\party{\Psi^*}{x^\mu} +2\Psi \Psi^* \party{A_\mu}{x^\mu}+ 2 A_\mu \Psi^* \party{\Psi}{x^\mu} \rbr=0 &\Rightarrow
\end{align*}

\begin{align}
  \notag &g^{\mu\nu} \party{}{x^\mu} \lbr   \Psi\party{}{x^\nu}\Psi^*  - \Psi^* \party{}{x^\nu} \Psi \rbr - g^{\mu\nu} \frac{iq}{\hbar c} \party{}{x^\mu}\lbr 2\Psi A_\mu \Psi^* \rbr=0 &\Rightarrow
  \notag\\
  \\& g^{\mu\nu} \party{}{x^\mu} \lsbr  \lbr \Psi\party{}{x^\nu}\Psi^*  - \Psi^* \party{}{x^\nu} \Psi \rbr - \frac{2iq}{\hbar c} \lbr \Psi A_\mu \Psi^* \rbr \rsbr = 0
  \label{befin}
\end{align}
\vspace{0.5cm}

Λόγω των εξισώσεων (\ref{dianisma}) και της (\ref{contineq2}) η (\ref{befin}) γίνεται : 
\vspace{0.5cm}
\begin{mdframed}
  \begin{equation}
    J^\mu =\frac{i \hbar}{2m} \lbr \Psi^* (\partial ^\mu \Psi)-\Psi(\partial ^\mu \Psi^*) \rbr - \frac{iq}{mc} A_\mu \Psi^* \Psi 
    \label{finqchar}
  \end{equation}
\end{mdframed}
\vspace{0.5cm}
Kαι τα τα δύο μέλη της (\ref{finqchar}) πολλαπλασιάστηκαν με $\frac{i \hbar}{2m}$. Δείχτηκε επομένως ότι για μια συνάρτηση που ικανοποιεί την εξίσωση \textlatin{Klein-Gordon} παρουσία διανυσματικού δυναμικού $A_\mu$, μ = 0,1,2,3, η εξίσωση συνέχειας ικανοποιείται από το τετραδιάνυσμα της σχέσης (\ref{finqchar}). 


