\section{Ειδική Σχετικότητα}

Στο παρόν κεφάλαιο θα δοθούν, συνοπτικά, κάποιοι ορισμοί της ειδικής θεωρίας της σχετικότητας που χρησιμοποιούνται σε επόμενα κεφάλαια με έμφαση, όμως, στο  πως μετασχηματίζονται θεμελιώδεις έννοιες του Ηλεκτρομαγνητισμού στα πλαίσια της θεωρίας αυτής. Επίσης θα εξηγηθούν και ορισμένοι συμβολισμοί. 

\subsection{Τανυστές}


Βασικό μαθηματικό εργαλείο αποτελούν οι τανυστές, που είναι μία γενίκευση των βαθμωτών και διανυσματικών μεγεθών. Υπάρχουν τρία είδη τανυστών :

\begin{itemize}
\item οι συναλλοίωτοι: $ A'^{ij} = \party{x'_i }{x_k} \party{x'_j}{x_l} A^{kl} $
\item οι ανταλλοίωτοι :$ C'_{ij} = \party{x_k}{x'_i} \party{x_l}{x'_j}C_{kl}$ 
\item οι μεικτοί τανυστές: $ B'^i_j  =\party{x'_i }{x_k} \party{x_l}{x'_j} B^l_k  $
\end{itemize}

\flushleft 
Κάποιες βασικές και χρήσιμες ιδιότητες , που προκύπτουν σαν άμεση συνέπεια από τα παραπάνω είναι ότι :

\begin{itemize}
\item Αν ένα τανυστής μηδενίζεται σε ένα σύστημα συντεταγμένων , μηδενίζεται σε όλα 
\item Αν δύο τανυστές είναι ίσοι σε κάποιο σύστημα συντεταγμένων τότε θα ισούνται σε όλα 
\item Από τις δυο προηγούμενες ιδιότητες, προκύπτει ότι η μορφή μιας τανυστικής εξίσωσης δεν αλλάζει με την αλλαγή συστήματος 
\end{itemize}
	
Οι πράξεις ανάμεσα σε τανυστές είναι: 

\begin{itemize}
\item Άθροισμα : $C^{ab}_c = \alpha A^{ab}_c +\beta B^{ab}_c $ 
\item Εξωτερικό γινόμενο : ${C^{a}_{bcd}}^e = \alpha A^{a}_b +\beta {B_{ab}}^{e} $ 
\item Συστολή :$ {T_{aq}}^{cq} = \sum\limits_{q=1}^n {T_{aq}}^{cq} $, κατα συνεπεια ο τανυστής $ T^a_a = T $ είναι βαθμωτό μέγεθος.  \
\item Εσωτερικό γινόμενο 
\end{itemize}

\subsection{Χωρόχρονος \textlatin{Minkowski} }

Στην ειδική θεωρία της σχετικότητας ο χώρος δεν είναι ο τρισδιάστατος ευκλείδειος χώρος αλλά 
ο τετραδιάστατος χωρόχρονος \textlatin{Minkowski}. Ο μετρικός τανυστής για το χώρο αυτό ορίζεται ως 

\begin{equation}
  g_{\mu,\nu} =
  \begin{pmatrix}
    g_{0,0}  & \cdots & g_{0,3} \\
    \vdots & \ddots & \vdots \\
    g_{3,0}  & \cdots & g_{3,3}
  \end{pmatrix}
  =
  \begin{pmatrix}
    1 & 0 & 0 & 0 \\
    0 & -1 & 0 & 0 \\
    0 & 0 & -1 & 0 \\
    0 & 0 & 0 &-1
  \end{pmatrix}
\end{equation}

Ο πίνακας που δίνει το  μετασχηματισμό ενός οποιουδήποτε τετραδιανύσματος ανάμεσα σε δύο αδρανειακά συστήματα, (μετασχηματισμοί \textlatin{Lorentz}) είναι : 

\begin{equation}
  \Lambda_\mu ^\nu=
  \begin{pmatrix}
    \Lambda^0 _0  & \cdots & \Lambda^0 _3 \\
    \vdots & \ddots & \vdots \\
    \Lambda^3 _0  & \cdots & \Lambda^3 _3
  \end{pmatrix}
  =
  \begin{pmatrix}
    \gamma & -\gamma \beta& 0 & 0 \\
    -\gamma \beta & \gamma & 0 & 0 \\
    0 & 0 & 1 & 0 \\
    0 & 0 & 0 & 1
  \end{pmatrix} 
\end{equation}


