\section{Ειδική Σχετικότητα}

Στο παρόν κεφάλαιο θα δοθούν, συνοπτικά, κάποιοι ορισμοί της ειδικής θεωρίας της σχετικότητας που χρησιμοποιούνται σε επόμενα κεφάλαια με έμφαση, όμως, στο  πως μετασχηματίζονται θεμελιώδεις έννοιες του Ηλεκτρομαγνητισμού στα πλαίσια της θεωρίας αυτής. Επίσης θα εξηγηθούν και ορισμένοι συμβολισμοί. 

\subsection{Τανυστές}


Βασικό μαθηματικό εργαλείο αποτελούν οι τανυστές, που είναι μία γενίκευση των βαθμωτών και διανυσματικών μεγεθών. Υπάρχουν τρία είδη τανυστών :

\begin{itemize}
\item οι συναλλοίωτοι: $ A'^{ij} = \party{x'_i }{x_k} \party{x'_j}{x_l} A^{kl} $
\item οι ανταλλοίωτοι :$ C'_{ij} = \party{x_k}{x'_i} \party{x_l}{x'_j}C_{kl}$ 
\item οι μεικτοί τανυστές: $ B'^i_j  =\party{x'_i }{x_k} \party{x_l}{x'_j} B^l_k  $
\end{itemize}

Κάποιες βασικές και χρήσιμες ιδιότητες , που προκύπτουν σαν άμεση συνέπεια από τα παραπάνω είναι ότι :

\begin{itemize}
\item Αν ένα τανυστής μηδενίζεται σε ένα σύστημα συντεταγμένων , μηδενίζεται σε όλα 
\item Αν δύο τανυστές είναι ίσοι σε κάποιο σύστημα συντεταγμένων τότε θα ισούνται σε όλα 
\item Από τις δυο προηγούμενες ιδιότητες, προκύπτει ότι η μορφή μιας τανυστικής εξίσωσης δεν αλλάζει με την αλλαγή συστήματος 
\end{itemize}
	
Οι πράξεις ανάμεσα σε τανυστές είναι: 

\begin{itemize}
\item Άθροισμα : $C^{ab}_c = \alpha A^{ab}_c +\beta B^{ab}_c $ 
\item Εξωτερικό γινόμενο : ${C^{a}_{bcd}}^e = \alpha A^{a}_b +\beta {B_{ab}}^{e} $ 
\item Συστολή :$ {T_{aq}}^{cq} = \sum\limits_{q=1}^n {T_{aq}}^{cq} $, κατα συνεπεια ο τανυστής $ T^a_a = T $ είναι βαθμωτό μέγεθος.  \
\item Εσωτερικό γινόμενο 
\end{itemize}

\subsection{Χωρόχρονος \textlatin{Minkowski}}

Στην ειδική θεωρία της σχετικότητας ο χώρος δεν είναι ο τρισδιάστατος ευκλείδειος χώρος αλλά 
ο τετραδιάστατος χωρόχρονος \textlatin{Minkowski}. Ο μετρικός τανυστής για το χώρο αυτό ορίζεται ως 

\[
  g_{\mu,\nu} =
  \begin{pmatrix}
    g_{0,0}  & \cdots & g_{0,3} \\
    \vdots & \ddots & \vdots \\
    g_{3,0}  & \cdots & g_{3,3}
  \end{pmatrix}
  =
  \begin{pmatrix}
    1 & 0 & 0 & 0 \\
    0 & -1 & 0 & 0 \\
    0 & 0 & -1 & 0 \\
    0 & 0 & 0 &-1
  \end{pmatrix}
\]

Παρακάτω δίνονται τα τετραδιανύσματα θέσης, ορμής, δυναμικού καθώς και η κλίση σε τέσσερις διαστάσεις:
\begin{align*} 
  x &=\{x,y,z,ict\} 
  \\p &=\{p_x,p_y,p_z,i\frac{E}{c}\}
  \\A &=\{A_x,A_y,A_z,iA_0\}
  \\ \nabla &= \{\party{}{x},\party{}{y},\party{}{z},iA_0\}
\end{align*} 
Το διάνυσμα που αντιπροσωπεύει την θέση ενός σωματιδίου γράφεται,με την βοήθεια του μετρικού τανυστή, για τις συναλλοίωτες και ανταλλοίωτες συντεταγμένες αντοίστιχα: 
\begin{align}
  \notag &x_{\mu}=g_{\mu \nu} x^\nu = \{ct,-x,-y,-z\} =\{x_0,x_1,x_2,x_3\}
  \\&x^{\mu}=g^{\mu \nu} x_\nu = \{x^0,x^1,x^2,x^3\}
  \label{dianisma}
\end{align}

Επομένως το εσωτερικό γινόμενο δύο διανυσμάτων θέσης θα είναι:
\[
  x \cdot x =x^\mu x_\mu =x^0 x_0 + x^1 x_1 + x^2 x_2 +x^3 x_3 = ct^2 -x^2-y^2-z^2  
\]

Για την τετραορμή εχουμε: $p^\mu =\{E/c,p_x,p_y,p_z\} $ οπότε για το γινόμενο των ορμών δύο σωματιδίων :

\begin{equation}
  p_1 \cdot p_2 = {p_1}^\mu {p_2}_\mu=\frac{E_1}{c} \frac{E_2}{c}-\vec{p_1}\cdot \vec{p_2} 
  \label{genrel2}
\end{equation}

Ενώ για το εσωτερικό γινόμενο θέσης-ορμής : 
\[
  x\cdot p= x^\mu p_\mu =x_\mu p^\mu= Et-\vec{x}\vec{p}
\]

Ο πίνακας που δίνει το  μετασχηματισμό ενός οποιουδήποτε τετραδιανύσματος ανάμεσα σε δύο αδρανειακά συστήματα, (μετασχηματισμοί \textlatin{Lorentz}) είναι : 

\begin{equation}
  \Lambda_\mu ^\nu=
  \begin{pmatrix}
    \Lambda^0 _0  & \cdots & \Lambda^0 _3 \\
    \vdots & \ddots & \vdots \\
    \Lambda^3 _0  & \cdots & \Lambda^3 _3
  \end{pmatrix}
  =
  \begin{pmatrix}
    \gamma & -\gamma \beta& 0 & 0 \\
    -\gamma \beta & \gamma & 0 & 0 \\
    0 & 0 & 1 & 0 \\
    0 & 0 & 0 & 1
  \end{pmatrix} 
\end{equation}

\subsection{Ειδική θεωρία της σχετικότητας και ηλεκτρομαγνητισμός }
Παρακάτω δίνοται οι σχέσεις μετασχηματισμού για το μαγνητικό και ηλεκτρικό πεδίο:

\begin{align*} 
  E'_x &= E_x        & B'_x &= B_x   
  \\E'_y &= \gamma(E_y-u B_z)         & B'_y &= \gamma \lbr B_y+\frac{u}{c^2}E_z \rbr  
  \\E'_z &= \gamma(E_z+u B_y)        & B'_z &= \gamma \lbr B_z+\frac{u}{c^2}B_z \rbr   
\end{align*} 

Για να βρεθεί η σχέση μετασχηματισμού κάποιου μεγέθους (πχ. ταχύτητα, θέση) αρκεί να υπολογιστεί η δράση του παραπάνω πίνακα στο μέγεθος. Ο μετασχηματισμός ενός τανυστικού μεγέθους δευτέρας τάξης θα είναι επομένως:
\[
k'^{\mu \nu}= {\Lambda^{\mu}}_{\lambda} {\Lambda^{\nu}}_{\sigma} k^{\lambda \sigma} 
\]
Εκτελώντας τις πράξεις το αποτέλεσμα είναι παρόμοιο με τους μετασχηματισμούς για το ηλεκτρομαγνητικό πεδίο. Το γεγονός αυτό οδηγεί στον ορισμό ενός τανυστικού μεγέθους του τανυστή πεδίου. 

\[
  F_\mu ^\nu=
  \begin{pmatrix}
    F^0 _0  & \cdots & F^0 _3 \\
    \vdots & \ddots & \vdots \\
    F^3 _0  & \cdots & F^3 _3
  \end{pmatrix}
  =
  \begin{pmatrix}
    0                &\frac{E_x}{c} &\frac{E_y}{c} & \frac{E_z}{c} \\
    -\frac{E_x}{c}   & 0            & B_z          & -B_y \\
    -\frac{E_y}{c}   & -B_z         & 0            & B_x \\
    -\frac{E_z}{c}   & B_y          & -B_x         & 0 \\
  \end{pmatrix} 
\]

Ο τανυστής αυτός,ουσιαστικά ενοποιεί το ηλεκτρικό και το μαγνητικό πεδίο και δίνει την δυνατότητα να γραφτούν οι εξισώσεις του \textlatin{Maxwell} σε συμπαγή μορφή τανυστικής εξίσωσης ως εξής: 

\begin{equation}
  \party{F^{\mu \nu}}{x^{\nu}}=\mu_0 J^{\nu}
  \label {contineq2}
\end{equation}

Για το δυναμικό ισχύει: 
\[
  F^{\mu \nu}=\party{A^\nu}{x_{\mu}}-\party{A^\mu}{x_{\nu}}
\]

Η πυκνότητα φορτίου $\rho = \lbr \frac{Q}{u} \rbr$και ρεύματος $(\vec{J}=\rho \vec{u}) $ μετασχηματίζονται σύμφωνα με τις σχέσεις που δίνονται παρακάτω. 

\begin{align}
  \rho &=\rho_0\frac{1}{\sqrt{1-\frac{u^2}{c^2}}}  & \vec{J}& =\rho_0\frac{\vec{u}}{\sqrt{1-\frac{u^2}{c^2}}}
\end{align}

Οι σχέσεις αυτές είναι παρόμοιες με τις σχέσεις μετασηματισμού των συσνιστωσών ενός τετραδιανύσματος του χωρόχρονου \textlatin{Minkowski}, το οποίο είναι : $J^{\mu}=(c\rho,J_x,J_y,Jz)$. 
Η εξισωση συνέχειας γίνεται: 
\begin{equation}
  \nabla \cdot J = -\party{\rho}{t} \Rightarrow \sum\limits_{i=1}^3 \party{J^i}{x^i}=\frac{1}{c}\party{J^0}{x^0} \Rightarrow \party{J^{\mu}}{x^{\mu}}=0
  \label{contineq}
\end{equation}

  
